\maketitle
%\thispagestyle{fancy}

\begin{abstract}

  Censorship in blockchain transaction processing systems occurs when an adversary intentionally arranges for exclusion of transaction items.
  %
  A design goal of public blockchains is that the cost of making such arrangements be as high as possible.
  %
  Unfortunately, quantifying this cost is complicated by the fact that the incentives of the participants depend on a panoply of hidden factors and the specifics of the censorship scenario.

  In this paper, we argue that the data needed to make sense of the cost of censorship by the most elementary strategy, which we call \emph{contracts for exclusion}, is captured by the structure of a strategic-form scheduling mechanism with the schedulers as players.
  %
  The strength of this approach is that it allows us to make statements about censorship resistance of any competitive scheduling environment --- including under novel proposals such as inclusion lists and multiple concurrent proposers --- on an equal footing.

  Given any scheduling mechanism, we formulate the CFE game and find minimal offers that achieve censorship with given probability $p$ in terms of invariants of the underlying mechanism.
  %
  For the case $p=1$ we also find conditions under which the cost of censorship so scales linearly in the number of schedulers.
  
\end{abstract}

\section{Introduction}

In blockchain systems, censorship is characterised as a deliberate attempt to prevent or delay a transaction or class of transactions from being posted to the chain \cite{wahrstatter2024blockchain}.
%
Broadly speaking, a blockchain is \emph{censorship resistant} if censorship is `hard' to achieve.
%
Censorship resistance is widely considered a key feature of permissionless public blockchains, especially for the functioning of financial primitives \cite{buterin2015problem}.

In principle, the censorship resistance of a blockchain can be quantified as the minimum cost to achieve censorship by any means.
%
Because of the diversity of available means --- which may target various layers of the technology stack and have different domains of effectiveness --- the estimation of these costs in general may pose a substantial challenge.
%
To attempt this calculation, we must distinguish and study the costs associated with each censorship strategy in context.

At a high level, every censorship strategy must proceed by arranging that sufficiently many of the parties responsible for deciding which items get included into the chain --- in this paper, we call them \emph{schedulers} --- either cannot or choose not to accept the targeted items.
%
Of these, arranging that a scheduler \emph{chooses} not to accept is a problem of manipulating economic incentives, the subject of this paper.
%


\paragraph{Contracts for exclusion}
%
Of all the strategies to manipulate the incentives of a scheduler so as to make handling some targeted items less attractive, perhaps the easiest to describe is to present him with a \emph{contract for exclusion}, that is, an agreement that triggers a value transfer depending on whether or not he accepts targeted items.
%
To be acceptable to a rational scheduler, the difference between the transfers in the two cases must compensate him for the profit he forgoes by omitting the items.
%
The cheapest possible such contract is one that \emph{only} requires him to censor $T$, and does not exclude any other potentially lucrative opportunity (see \S\ref{section:replacement}).

While the obvious interpretation of such a contract is as a bribe that is paid if the scheduler agrees to censor, the theory also applies to threats --- such as the notorious OFAC sanctions \cite{wright2022defi} --- where a cost imposed on the scheduler if he does handle the targeted items.
%
These two approaches differ by the payment amounts and directions, but not on the condition.

We focus on this case because it is universal, relatively tractable, and because it has been recently been used to motivate novel scheduling mechanisms in the Ethereum chain \cite{FPR,burian2024censorship}.




\paragraph{Defining censorship cost}
%
A rational scheduler will not enter into a contract for exclusion unless the terms of the contract provide that he be compensated for his opportunity cost in sitting out of the market.
%
Exactly how much compensation is required, and hence the cost to an adversary of offering such terms, depends essentially on the fine details of the environment and the decision problem it presents to the scheduler:
%
\begin{enumerate}

  \item 
    The mechanics of the scheduling pipeline, that is, how decisions about blockspace allocation are finalised;
  \item 
    Details of the targeted items such as the maximum fee;
  \item
    The variable costs, preferences, and beliefs of the scheduler, his view of other opportunities that complement or conflict with the targeted items, and his capacity to exploit them.
\end{enumerate}
%
Moreover, the third item --- the scheduler's \emph{type} --- is generally not known to the adversary, who therefore cannot compute ex-ante the minimum compensation the scheduler would accept.
%
Instead, the adversary's beliefs about his counterparty's type forms some probability distribution, and the best he can do is make an offer that succeeds with probability exceeding some threshold $0<p<1$ of his choosing.

While the system planner may have agency over some of the details of the scheduling algorithm, in general he cannot directly control the other two items.
%
Thus, it is not possible to describe the cost of censorship associated to a particular design choice, or even a design choice plus a transaction item, by a single number: it is a function over the product of the unit interval, parametrising the threshold, and a potentially complicated space of scheduler types.




\paragraph{Multiple schedulers}
%
In public blockchains, the responsibility of deciding which transactions to add to the schedule rotates among multiple independent schedulers from epoch to epoch.
%
Assuming transactions cannot be sequenced multiple times, these schedulers are effectively in competition with each other for order flow.
%
An adversary that wishes to censor transactions for multiple epochs must therefore make arrangements with multiple competing schedulers.
%
Estimating the cost of doing so must therefore take into account the types of these various schedulers and their strategic interactions.

There are also systems that allow multiple schedulers to participate in constructing the schedule for a single epoch:
%
\begin{itemize}

  \item 
    In Filecoin's expected consensus system, a random number of consensus nodes are elected each epoch to construct a block; the resulting blocks are concatenated and deduplicated.

    Similar ideas, based on merging the blocks produced by multiple consensus nodes, have started to appear in the Ethereum context \cite{FPR,burian2024censorship}.

  \item
    In Ethereum's proposer-builder separation \cite{thegostep2021mevboost}, multiple builders construct tentative schedules and compete in a single item auction to have their schedule committed.

  \item
    So-called inclusion lists impose constraints on blocks to force the inclusion of certain transactions, allowing the creators of these lists to act as schedulers in addition to the builder of the main block \cite{thomas2024forkchoice}.

\end{itemize}
%
Some of these ideas have been introduced with the explicit goal of improving censorship resistance; for such proposals, therefore, it would be very useful to have a framework to quantify their effect.

In this paper, we argue that the full details of scheduling environments with multiple schedulers whose types may differ are captured by the formalism of mechanism design \cite{milgrom2004putting}.
%
In this formalism, the scheduling algorithm, value of the targeted items, and private information of the scheduler, are captured by the outcome function, payoff functions, and type distributions of a strategic form mechanism with schedulers as players.

If we endow the strategy spaces $\Action$ of our mechanism with a distinguished point $\bot$ representing `do nothing,' then a contract for exclusion adjusts incentives by introducing a value transfer to each player who selects $\bot$ as their strategy.
%
The cost of censorship at chance bound $p\in[0,1]$ is then a numerical invariant of the (pointed) scheduling mechanism.



\begin{example}

  The rotating scheduler mechanism with distinct schedulers at every time step can be modelled as sequentially arriving short-lived bidders; see \cite[Chap.~2]{gershkov2014dynamic}.

\end{example}

By couching the scheduling environment as a mechanism design problem, we also give ourselves easy access to an inventory of standard auction formats for introducing competition at the scheduler level.
%
With the flexibility of being able to reason about censorship under arbitrary mechanisms, we can do unrestricted optimisation over the mechanism design space, selecting on other properties of interest such as incentive compatibility or credibility.



\subsection{Results}
\paragraph{Bounding censorship resistance}

In light of the preceding discussion, we introduce the following definition:

\begin{definition}[Cost of censorship]

  The \emph{cost to censor} $\CC{p}(\Mechanism)$ a blockspace supply mechanism $\Mechanism$ at chance bound $p\in[0,1]$ is the smallest total amount that can be offered to schedulers in order to achieve an overall probability of at least $p$ that the target items are excluded.

\end{definition}




\begin{example}[Single scheduler, posted price]

  Suppose $T$ is a single transaction with transaction fee $\tau$, and let there one scheduler having marginal cost $\tilde{\theta}>0$ for handling $T$.
  %
  If the scheduler handles $T$, he receives the full fee payment $\tau$, for a profit of $\tau-\tilde{\theta}$.
  %
  The lowest offer that would be accepted by the scheduler with probability at least $p$, and hence the cost to censor at threshold $p$, is $\CC{p}=\tau+Q_{-\tilde{\theta}}(p)$, where $Q_{\tilde{\theta}}:[0,1]\rightarrow\R$ is the quantile function of $\tilde{\theta}$.

\end{example}


\begin{example}[Multiple schedulers, posted price]

  Let $T$ be a single transaction with fee limit $\tau$ and posit $N$ schedulers, each with vanishing marginal costs.
  %
  Suppose that each scheduler makes an independent decision about whether to declare himself `in the market' for handling $T$.
  %
  Then, if any scheduler makes such declaration, some mechanism selects one and awards him the transaction fee, and the transaction is included; otherwise it is not.

  If exactly one scheduler declares himself active, then he stands to gain a sure monopoly profit of $\tau$.
  %
  Hence, if a scheduler sees that a censorship attempt is ongoing and believes his competitors will accept their offers and sit out, then he will accept the offer presented to him only if it beats his expected monopoly profit of $\tau$.
  %
  A censoring adversary must therefore offer at least $\tau$ to each of the $N$ schedulers, putting the total cost of censorship (at $p=1$) at $N\tau$.
  
  This is the intuition behind the \emph{product law}, which poses that the cost of censorship is linear in the number of schedulers and the transaction fee limit.
  %
  See Theorem \ref{thm:product-law} for a precise statement.

\end{example}

\begin{example}[Multiple schedulers, second price auction]

  Consider a setting like the previous example, except that the mechanism that selects a scheduler to handle the item is a second price procurement auction.
  %
  That is, each scheduler puts up a \emph{quote} $f\leq \tau$ to handle the item, and the scheduler who offers the lowest quote handles the item, with transaction fee equal to the second lowest quote or, if only one scheduler participates, the `reservation price' $\tau$.
  %
  Since each scheduler has vanishing costs, their dominant strategy (in the absence of censorship bribes) is to quote $f=0$; the transactor then pays $\tau$ if exactly one scheduler quotes, and $0$ if more than one does.
  %
  By the same logic as before, the cost of censorship is $N\tau$.
  
  The outcomes are quite similar to those of the 'conditional tipping` system introduced in \cite{FPR}; the perspective of this paper is outcomes of this form are actually a natural consequence of competition among schedulers, and do not need to be explicitly programmed into the definition of the mechanism.

\end{example}

In order to state our general results on $\CC{p}$, it is necessary to dip our toes into some of the details of the model.
%
Suppose an adversary $\Adversary$ makes an offer $z_i\geq 0$ to each scheduler $i$ in a population $\Population$ to sit out of the scheduling mechanism.
%
Each scheduler accepts this offer if and only if $z_i$ exceeds $i$'s value function $\Util_i(\tilde\theta_i)$, that is, his expectation of the payoff he could derive by participating given his type $\tilde\theta_i$.

From $\Adversary$'s perspective, $i$ accepts with probability $\lambda_i=\Prob[\Util_i(\tilde\theta_i)\leq z_i] = F_{\Util_i(\tilde\theta_i)}(z_i)$.
%
The probability of successful censorship is the probability that all schedulers accept their offers, which is $p=\prod_i\lambda_i$ if these events are independent.
%
In this case, we say that the offer $z\in\uR^\Population$ is \emph{$p$-effective}.
%
Then $\CC{p}(\Mechanism)$ is simply the infimum of $\sum_iz_i$ over the set of $p$-effective offers $z$.

Now, $i$'s estimation of his payoffs naturally depend on whether his competitors participate or sit out of the mechanism.
%
A subtlety therefore emerges if scheduler $i$ is aware that a censorship attempt is going on --- which, given that he himself has received an offer of reward for abstention, is reasonable to suspect --- and hence that his competitors are likely to have also received an offer to abstain.
%
This complicates the estimation of $\lambda_i=\Prob[\Util_i > z_i]$, which now also depends on $\lambda_j$ for $j\neq i$, and hence in turn on the bribes $z_j$ offered to $i$'s competitors.
%
The calculation now depends on identifying a Nash equilibrium.

In this paper, as in \cite{FPR}, we mainly consider the case where the bribe offered to all schedulers is \emph{common knowledge} (but see \S\ref{section:asymmetry}).
%
In this case, we are able to construct $\lambda$-effective Nash equilibria for any `abstention vector' $\vec{\lambda}\in[0,1]^\Population$ of acceptance probabilities.

\begin{theorem}[Corollary \ref{thm:equilibrium-from-quantile}]

  For any truthful mechanism $\Mechanism$ and any abstention profile $\bar\lambda\in[0,1)^\Population$, there exists an offer $z\in\uR^\Population$ and a Bayes-Nash equilibrium of the abstention game $\Mechanism(z)$ with abstention profile dominating $\lambda$. 
  %
  Moreover, $z$ can be chosen minimal among such offers.
  
  Furthermore, there exists a sequence of functions $Z_i(\tilde\theta_i,\lambda_{-i})$ depending only on $\Mechanism$ which:
  %
  \begin{itemize}
    \item
      are polynomial of degree $N-1$ in the $\lambda_{-i}$;
    \item
      have coefficients expressible as sums of value functions $\Util_{i,S}(\tilde\theta_i)$ of the underlying mechanism conditioned on subsets $S\subseteq\Population$ remaining active;
  \end{itemize}
  %
  such that in the aforementioned Nash equilibrium, scheduler $i$ abstains, resp.~remains active, whenever
  %
  \[
    z_i > Z_i(\tilde\theta_i,\lambda_{-i}) \qquad \text{resp.} \qquad z_i < Z_i(\tilde\theta_i,\lambda_{-i}).
  \]
  
\end{theorem}

The polynomials $Z_i(\tilde\theta_i,\lambda_{-i})$ are defined by an explicit formula in \S\ref{section:dsic}.
%
The minimal offer can be constructed as the $\lambda_i$-quantile of $Z_i$.

\begin{corollary}[Upper bounds for the cost of censorship]

  For any $\lambda\in[0,1]^\Population$,
  %
  \[
    \CC{\prod_{i=1}^N\lambda_i}(\Mechanism) \leq \sum_{i=1}^N Q_{Z_i(\tilde\theta_i,\lambda_{-i})}(\lambda_i) .
  \]
  %
  For any $p\in[0,1]$, we have
  %
  \[
    \CC{p}(\Mechanism) = \inf_{\prod_i\lambda_i \geq p} \left(\sum_{i=1}^N Q_{Z_i(\tilde\theta_i,\lambda_{-i})}(\lambda_i) \right).
  \]
  

\end{corollary}


Since upper bounding the cost of censorship is as simple as constructing effective offers, this gives us a machine to construct a lot of upper bounds.
%
As protocol designers or transaction originators, however, we might be more interested in establishing \emph{lower} bounds.
%
In the special case $p=1$, the infimum collapses to a single element.

\begin{corollary}[Sure censorship]

  We have $\CC{1}(\Mechanism) = \sum_{i=1}^N Q_{Z_i(\tilde\theta_i,1)}(1)$.

\end{corollary}

For $p<1$, we may wonder whether we can restrict attention to a subset of the simplex $\prod_{i\in\Population}\lambda_i=p$, such as the diagonal $\lambda_i = p^{1/N}$.
%
Unfortunately, even in the symmetric IPV case that scheduler types are i.i.d, this turns out not to be the case (\S\ref{section:asymmetry}), and constructing lower bounds requires consideration of arbitrary vectors of acceptance thresholds.

Nonetheless, if the conditional payoffs $\Util_{i,S}(\Mechanism)$ of the scheduling mechanisms are computationally tractable and we can sample efficiently from the distribution of $\tilde\theta_i$, then the quantile functions $Q_{Z_i}$ can be estimated using Monte Carlo methods and the infimum approximated via a grid search.

\paragraph{Scaling censorship resistance}

The rotating scheduler architecture of public blockchains such as Bitcoin and Ethereum is generally thought to increase the difficulty of censoring transactions over a period of many blocks.
%
If we allow ourselves to assume that:
%
\begin{enumerate}
  \item Every block over the period of interest is made by a different scheduler;
  \item an adversary may enter into contracts with schedulers as they arrive, in time for the scheduler to make decisions about item inclusion;
\end{enumerate}
%
then it is quite easy to construct an informal argument as to why the cost of censorship scales linearly in the number $N$ of blocks: since in this case, each scheduler's payoff depends only on its decision and the actions of those that came before it, equilibrium can be computed by solving the decision problem at each block in order, i.e.~without needing to take into account strategic play.
%
If the marginal value of the item is simply a time-invariant transaction fee $f$, the cost to censor in this fashion is therefore $fN$; compare \cite[Ex.~2]{FPR}.

Based on similar intuition, recent proposals \cite{FPR,thomas2024forkchoice} have sought to improve the single-slot censorship resistance (or `neutrality' \cite{ma2024uncrowdable}) properties of the Ethereum chain by increasing the number of agents capable of allocating blockspace in each slot.

If we restrict ourselves to the limited context of zero failure tolerance and mechanisms in which the scaling behaviour of scheduler payoff can be computed in terms of a simple `fee limit' $\tau$, we find the linear scaling rule holds abstractly:

\begin{theorem}[Product law for sure censorship]
  \label{thm:product-law}

  Let $\Mechanism_\tau$ be a family of efficient single item auctions with reservation price $\tau$.
  %
  Suppose schedulers costs $\tilde\theta_i$ are i.i.d, bounded below, and independent of $\tau$.
  %
  Then $\CC{1}(\Mechanism_\tau)<\infty$, and for sufficiently large $\tau_0\gg 0$,
  \[
    \frac{\partial}{\partial\tau}\CC{1}(\Mechanism_\tau) \equiv N
  \]
  %
  where $N$ is the number of schedulers.

  If there exist schedulers with arbitrarily small costs, then $\CC{1}(\Mechanism_\tau) = N\tau$.
  
\end{theorem}

Are there any general laws for how censorship cost scales with the number of active schedulers?
%
Do the laws extend to scenarios with uncertainty about scheduler types and where positive censorship failure rate is tolerated?
%
I don't know, but in this paper we find that calculations in specific scenarios can be surprisingly tractable.

\subsection{Previous work}

The importance to blockchains of having multiple independent entities able to produce blocks and vote on the canonical chain is, of course, already recognised in the Bitcoin whitepaper.
%
On the the effect of miner count on timely and reliable transaction inclusion, Nakamoto writes: \emph{New transaction broadcasts do not necessarily need to reach all nodes. As long as they reach many nodes, they will get into a block before long.} \cite[\S5]{nakamoto2008bitcoin}.
%
Attempts to quantify this property in the Bitcoin literature focused on the latency of transaction inclusion in the presence of network failures \cite{pappalardo2018blockchain} without explicitly considering intentional censorship.

Although the early Bitcoin community was undoubtedly aware of the threat of deliberate censorship as well as benign network failures, the specific attack vector of targeted transaction exclusion by cabals of miners does not appear to have attracted much discussion unti the 2015 post of Buterin \cite{buterin2015problem}.
%
This post also highlighted the indispensability of the censorship resistance property, including over short periods, for the proper functioning of financial primitives such as loans.

Subsequently, academia has picked up this thread of reasoning and made the link between decentralisation of the scheduler set and censorship resistance \cite{gencer2018decentralization,silva2020impact}, and especially in the wake of the OFAC sanctions of Tornado Cash \cite{wahrstatter2024blockchain}.
%
Discussion of the multiplicity of approaches to censorship appears in \cite[\S4.3]{wahrstatter2024blockchain}.

Outside of the blockchain literature, some scattered work has considered censorship in other types of information systems such as file sharing networks \cite{perng2005censorship,danezis2004economics}.



%%%%%%%%%%%%%%%%%%%%%%%%%%%%%%%%%%%%%%%%%%%%%%%%%%%%%%%%%%%%%%%%%%%%%%%%%%%%%%%

\section{Model}
\label{section:model}

We introduce the following objects:
%
\begin{itemize}
  \item 
    A population $\Population$ of \emph{schedulers}, which for convenience we sometimes identify with the set $\{1,\ldots,N\}$;
  \item 
    For each scheduler $i\in\Population$, a \emph{type space} $\Theta_i$.
    
    In examples, we will focus on schedulers with a single type parameter $\theta_i\in\uR$ representing the marginal cost to handle the targeted items.
  \item
    A commonly known distribution $F_{\tilde\theta}$ over the type space $\Theta=\prod_{i\in\Population}\Theta_i$.
    %
    A $\Theta_i$-valued random variable having this distribution is written $\tilde\theta_i$.
    %
    We always assume that the $\tilde\theta_i$ are independent.
  \item
    For each $i\in\Population$, a strategy space $\Action_i$ with a distinguished point $\bot_i$ representing \emph{abstention};
  \item
    \emph{Reward functions} $U_i:\Action \rightarrow \R$ for each $i\in\Population$;
  \item
    Given a strategy profile $s\in\Action$, the \emph{active set} 
    \[
      \Population_\Active(s) := \{i\in\Population \mid s_i \neq\bot_i\} 
    \] 
    is the set of schedulers that do not abstain.
    %
    For every subset $S\subseteq\Population$ and $i\in S$, a selected solution $s^*_{i,S}:\Theta_i\rightarrow\Action_i$ such that $s_S^*(\theta)\in\prod_{i\in S}\Action_i$ defines a Bayes-Nash equilibrium among the players in $S$ given that $\Population_\Active = S$.
    
    If $\Mechanism$ is DSIC, these don't depend on $S$.

\end{itemize}
%
%Details of the coherence conditions required on the distinguished points $\bot_i$ are discussed in the appendix.

After schedulers learn their types, but before they decide on a strategy, an adversary $\Adversary$ may offer each scheduler a direct payment $z_i\in\uR$ to \emph{abstain}, i.e.~choose the strategy $\bot_i$.
%
This defines a new mechanism $\Mechanism(z)$, with player populations, types, and strategy spaces as before, and value functions defined by the formula
%
\[
  U_{i,\Mechanism(z)}(s) = \left\{\begin{array}{ll} 
    U_{i,\Mechanism}(s) & s_i \neq \bot_i \\
    U_{i,\Mechanism}(s) + z_i & s_i = \bot_i
  \end{array}\right.
\]
%
We will be interested in Nash equilibria of the modified game, and especially the probability that all players choose abstention under such equilibria.



\subsection{Sure censorship}

We focus first on the case of \emph{sure censorship}, that is, $\CC{1}(\Mechanism)$.
%
By definition, $Z_i(\tilde\theta_i,1) = \Util_{i,\{i\}}(\theta_i)$ is $i$'s expected payoff given that he has the monopoly.
%
We therefore have
%
\[
  \CC{1}(\Mechanism) = \sum_{i\in\Population} \Util_{i,\{i\}}(\theta_i^*)
\]
%
where $\theta_i^*\in\argmax_{\supp(\tilde\theta_i)} \Util_{i,\{i\}}$ is a type that maximises monopoly payoffs.
%
In particular, if the support of the type distribution is compact and the monopoly payoff functions are continuous, then $\CC{1}(\Mechanism)$ is finite.



\begin{definition}[Efficient auctions with reservation price]

  A \emph{single item procurement auction} is a mechanism in which:
  %
  \begin{itemize}
    \item Type spaces are an interval in $\uR=[0,\infty)$, and represent the marginal cost of handling the item;
    \item The utility functions have the form
    \[
      U_i(s,\tilde\theta) = f(s) - I_{i = x(s)}\tilde\theta
    \]
    where $f:\Action\rightarrow \R^\Population$ defines a fee schedule (paid by the item source to the schedulers) and $x:\Action\rightarrow \Population$ selects a single scheduler to handle the item, and $I_{i=x(s)}=1$ when $i=x(s)$ and $0$ otherwise.
  \end{itemize}

  A single item procurement auction $\Mechanism$ is \emph{efficient with reservation price $\tau$} if the allocation rule as a function of types satisfies
  %
  \[
    x(\theta) \in \argmax\{\theta_i \mid i\in\Population\} 
  \]
  %
  whenever $\theta_i\geq \tau$ for some $i\in\Population$, otherwise $x(\theta) = \bot$ (i.e.~the item is not allocated).

\end{definition}

\begin{proof}[Proof of Theorem \ref{thm:product-law}]

  Using $\Util_{i,\{i\}}(\tau) = 0$ as a reference point, we apply Myerson's payoff equivalence theorem \cite[Thm.~3.3]{milgrom2004putting} to identify the payoff functions $\Util_{i,\{i\}}(\theta_i)$ with those of the second price auction with reserve $\tau$; that is, $\Util_{i,\{i\}}(\theta_i) = (\tau-\theta_i)_+$.
  %
  If $k\leq\tau $ is a greatest lower bound for $\tilde\theta_i$, then $\CC{1}=\sum_{i\in\Population}\Util_{i,\{i\}}(k) = N(\tau - k)$.
  %
  \qedhere

\end{proof}


\subsection{Solution in truthful case}
\label{section:dsic}

Let's suppose now that $\Mechanism$ is DSIC, so that in particular the selected strategic plans $s^*_{i,S}$ do not depend on the competitor set $S$.
%
This is going to allow us to restrict attention to strategies in which each scheduler either plays $s^*_i(\theta_i)$ or $\bot_i$, or a mixture thereof, depending on whether $z_i$ beats his expected payoff from participating.
%
The expected payoff from participating still depends on which competitors sit out.

A function $\lambda_i:\Theta_i\rightarrow[0,1]$ defines a strategic plan as a mixture of $s^*_i(\theta_i)\neq\bot_i$ and $\bot_i$ by picking out the probability with which $\bot_i$ is played.
%
For any such strategy, write
%
\[ 
  \bar\lambda_i := \Expectation[\lambda_i(\tilde\theta_i)] = \Prob[i\in\Population_\Active].
\]
%
If $\lambda=(\lambda_i)_{i\in\Population}$ is a profile of strategies of this form, we refer to $\bar\lambda = (\bar\lambda_i)_{i\in\Population}\in [0,1]^\Population$ as the \emph{abstention profile}.

For each $i\in S\subseteq\Population$, let 
\[
  \Util_{i,S}(\theta_i) := \Expectation[U_i(s^*(\tilde\theta) \wedge [S])\mid \tilde\theta_i=\theta_i]
\]
%
where for $s\in\Action$ and $S\subseteq\Population$ we write $s\wedge [S]$ for the `masked' strategy profile defined by 
%
\[
  (s\wedge[S])_j = s_j \text{ for } j\in S \qquad (s\wedge[S])_j = \bot_j \text{ for } j\in\Population\setminus S.
\]

Define a function
%
\[
  Z_i(\tilde\theta_i,\lambda_{-i}) := \sum_{i\in S\subseteq \Population} \Util_{i,S}(\tilde\theta_i)\prod_{j\in\Population\setminus S}\lambda_j\prod_{k\in S\setminus\{i\}}(1-\lambda_k).
\]
%
It is a deterministic function of the random variable $\tilde\theta_i$.
%
It has the following interpretation: if each scheduler $j\in\Population\setminus\{i\}$ plays a mixed strategy with abstention profile $\lambda_{-i}$, then
%
\begin{align*}
  Z_i(\theta_i,\Expectation[\lambda_{-i}(\tilde\theta_{-i})]) &= \sum_{i\in S\subseteq \Population} \Util_{i,S}(\theta_i)\prod_{j\in\Population\setminus S}\Prob[j\not\in\Population_\Active]\prod_{k\in S\setminus\{i\}}\Prob[k\in\Population_\Active] \\
  &= \sum_{i\in S\subseteq\Population}  \Util_{i,S}(\theta_i) \Prob[\Population_\Active = S] \qquad \text{by independence}\\
  &= \Expectation\left[ U_i(s_i,\lambda_{-i}(\tilde\theta_{-i}))\right].
\end{align*}
%
That is, $Z_i(\tilde\theta_i,\lambda_{-i})$ is scheduler $i$'s expected payoff from participating, given that his competitors play the strategy profile $\lambda_{-i}:\Theta_{-i}\rightarrow [0,1]^{\Population\setminus\{i\}}$.




\begin{proposition}[Characterisation of equilibria]
  \label{thm:bidder-equilibrium}

  A strategy profile $(\lambda_i)_{i\in\Population}$ is a Nash equilibrium of the abstention game $\Mechanism(z)$ if and only if
  %
  \[
    \lambda_i(\theta_i) = \left\{\begin{array}{ll} 
      1 & \text{if } z_i > Z_i(\theta_i,\Expectation[\lambda_{-i}(\tilde\theta_{-i})]) \\ 
      0 & \text{if } z_i < Z_i(\theta_i,\Expectation[\lambda_{-i}(\tilde\theta_{-i})])
    \end{array}\right.
  \]
  %
  for each $i\in\Population$.

\end{proposition}
%
\begin{proof}

  The payoff of a mixed strategy $\lambda\in[0,1]$ given competitors playing $\lambda_{-i}$ is precisely
  %
  \[
    \lambda z_i + (1-\lambda) Z_i(\theta_i,\Expectation[\lambda_{-i}(\tilde\theta_{-i})])
  \]
  %
  which is maximised at $1$ if $z_i>Z_i$, $0$ if $z_i<Z_i$, and is indifferent to $\lambda\in[0,1]$ when $z_i=Z_i$. \qedhere

\end{proof}

Using this result, we can construct a tight correspondence between Nash equilibria of this form and the abstention profile $\bar\lambda\in [0,1]^\Population$.
%
Indeed, the only ambiguity left by Proposition \ref{thm:bidder-equilibrium} is the strategy for types belonging to the \emph{indifference region}
%
\[  
  \Theta_i(z_i):=\{\theta\in\Theta_i \mid Z_i(\theta,\lambda_{-i}) = z_i\}
\]
%
of $z_i$.


\begin{corollary}[Uniqueness of equilibria from expected play]
  \label{thm:expected-equilibrium-uniqueness}

  A Nash equilibrium strategy profile is uniquely determined away from $\Theta_i(z_i)$ by its abstention profile.
  %
  If $Z_i(-,z)$ is strictly monotone increasing in its first argument, then it is uniquely determined everywhere.

\end{corollary}
%
\begin{proof}

  The first claim is immediate from Proposition \ref{thm:bidder-equilibrium}.
  %
  For the second, let $\theta_i\in\Theta_i$ be the unique type such that $\Util_i(\theta_i,z_{-i})=z_i$.
  %
  Then
  %
  \begin{align*}
    \bar\lambda_i  &= \Expectation[\lambda^*_i(\tilde\theta_i)] \\
    &= \lambda^*_i(\theta_i)\cdot\Prob[\tilde\theta_i=\theta_i] + \Prob[\tilde\theta_i>\theta_i]
  \end{align*}
  %
  which can be rearranged to give a formula for the only remaining value $\lambda^*_i(\theta_i)$. 
  %
  \qedhere

\end{proof}

By abuse of language, we will call a vector of probabilities $(\lambda_i)_{i=1}^N$ an \emph{equilibrium} if the inequalities of Corollary \ref{thm:expected-equilibria-existence} are satisfied.

\begin{corollary}[Existence of equilibria from expected play]
  \label{thm:expected-equilibria-existence}

  A vector $\lambda\in[0,1]^\Population$ is the abstention profile of a Nash equilibrium if and only if the inequalities
  %
  \[
    \Prob[Z_i(\tilde\theta_i,\lambda_{-i}) < z] \leq \lambda_i \leq \Prob[Z_i(\tilde\theta_i,\lambda_{-i}) \leq z]
  \]
  %
  are satisfied.

\end{corollary}
%
\begin{proof}
  
  Suppose given $\lambda$ satisfying the desired inequalities.
  %
  Proposition \ref{thm:bidder-equilibrium} gives us a candidate for $\lambda^*_i:\Theta_i\rightarrow[0,1]$ away from
  %
  \[  
    \Theta_i(z_i):=\{\theta\in\Theta_i \mid Z_i(\theta,\lambda_{-i}) = z_i\}.
  \]
  %
  It remains to construct a function $\lambda^*_i:\Theta_i(z_i)\rightarrow[0,1]$ such that
  %
  \begin{align*}
    \lambda_i &= \int_{\Theta}\lambda_i^*(\theta_i) dF_{\tilde\theta_i} \\
    &= \int_{\Theta_i(z_i)} \lambda_i^*(\theta)dF_{\tilde\theta} + \Prob[Z_i(\tilde\theta_i,\lambda_{-i}) < z],
  \end{align*}
  %
  that is, such that
  %
  \[
    \lambda_i - \Prob[Z_i(\tilde\theta_i,\lambda_{-i}) < z] = \int_{\Theta_i(z_i)} \lambda_i^*(\theta)dF_{\tilde\theta}
  \]
  %
  for each $i$.
  
  Since 
  %
  \[
    0 \leq \lambda_i - \Prob[Z_i(\tilde\theta_i,\lambda_{-i}) < z] \leq \Prob[Z_i(\tilde\theta_i,\lambda_{-i}) \leq z] - \Prob[Z_i(\tilde\theta_i,\lambda_{-i}) < z] ]= \Prob[\tilde\theta_i\in \Theta_i(z_i)]
  \]
  %
  this is clearly possible; for example, we could take $\lambda_i|_{\Theta_i(z_i)}$ to be the indicator function of any sub-event of $\Theta_i(z_i)$ having measure $\lambda_i - \Prob[Z_i(\tilde\theta_i,\lambda_{-i}) < z]$.
  %
  \qedhere

\end{proof}


\begin{corollary}[Reverse engineering equilibria]
  \label{thm:equilibrium-from-quantile}

  Let $\lambda\in[0,1]^\Population$ and write
  %
  \[
    z_i = Q_{Z_i(\tilde\theta_i,\lambda_{-i})}(\lambda_i) \quad i\in\Population.
  \]
  %
  Then there exists a Nash equilibrium of $\Mechanism(z)$ with abstention vector dominating $\lambda$.
  %
  Moreover, $z$ is the lowest offer that can achieve an abstention profile that dominates $\lambda$: if $z'\in\uR^\Population$ is any offer that admits an equilibrium $\lambda'$ with $\lambda_i'\geq\lambda_i$ for all $i$, then $z_i'\geq z_i$ for all $i$.

\end{corollary}
%
\begin{proof}

  By general properties of the quantile function. \qedhere

\end{proof}

\begin{corollary}[Upper bounds for the cost of censorship]

  For any $\lambda_i$,
  %
  \[
    \CC{\prod_{i=1}^N\lambda_i}(\Mechanism) \leq \sum_{i=1}^N Q_{Z_i(\tilde\theta_i,\lambda_{-i})}(\lambda_i) .
  \]
  %
  For any $p\in[0,1]$, we have
  %
  \[
    \CC{p}(\Mechanism) = \inf_{\prod_i\lambda_i \geq p} \left(\sum_{i=1}^N Q_{Z_i(\tilde\theta_i,\lambda_{-i})}(\lambda_i) \right).
  \]
  

\end{corollary}


















\section{Discussion}

We now discuss some alternative perspectives and extensions of the theory presented in \S\ref{section:model}.

\subsection{Censorship resistance}
\label{section:cr}

We have discussed the cost of censorship at a chance bound $p$, but not `censorship resistance' of a mechanism.
%
To make sense of censorship resistance as a function of the mechanism alone, we must eliminate the dependence on $p$.

Suppose a censoring adversary stands to gain a payoff of $v_\Adversary>0$ for successfully censoring items.
%
Then if $\Adversary$ chooses offers $z\in\uR^\Population$ so as to achieve censorship with probability $p$, he derives a utility of
%
\[
  U_\Adversary(z;\bar{\lambda}) = pv_\Adversary - \sum z \leq pv_\Adversary - \CC{p}(\Mechanism).
\]
%
If $\CC{p}(\Mechanism)/p > v_\Adversary$, then $\Adversary$ would rather do nothing, i.e.~offer $z=0$.

(We could have a situation where any offer that achieves censorship at threshold $p$ also achieves censorship with probability $p'>p$; in this case $\CC{p'}=\CC{p}$.)

This leads us to make the following definition:
%
\begin{definition}[Censorship resistance]

  A mechanism $\Mechanism$ is $K$-\emph{censorship resistant} if $\CC{p}(\Mechanism)/p>K$ for all $p\in(0,1]$.

\end{definition}
%
That is, $\Mechanism$ is $K$-censorship resistant if no rational, risk-neutral adversary with type $v_\Adversary\leq K$ would attempt censorship at any chance bound.
%
Unfortunately, we currently don't have many tools at our disposal to find lower bounds on $\CC{p}$ for all $p>0$.

\begin{example}

  Let $\Mechanism$ be the second price auction with $F_{\tilde\theta}$ concave for $\theta\geq Q_{\tilde\theta}(p_0)$.
  %
  Then $\CC{p} \geq N(\tau + Q_{-\tilde\theta}(p^{1/N})p^{\frac{N-1}{N}})$, and so
  %
  \[
    \CC{p}/p \geq N(\tau/p + Q_{-\tilde\theta}(p^{1/N})/p^{1/N})
  \]
  %
  for $p\geq p_0$.
  %
  If $Q_{-\tilde\theta}$ drops off sharply as $p$ goes below $1$, then this function may be minimised at $p<1$.

\end{example}

\begin{comment}


\paragraph{Common value case}
Here $\bar\lambda(z)=\lambda^*(z)$ and we can solve for $z$.
%
The special case $\tilde\theta \equiv 0$ is studied in \cite{FPR}.


\begin{example}[Second price common value auction]

  As we show in an earlier example, the surplus coefficients are $c_0 = v$ and $c_k=0$ for $k>0$.
  %
  If $z>v$, all bidders accept the bribe.
  %
  If $z\in[0,v]$, the bidders pay a mixed strategy defined by the equation $z=v\lambda^{N-1}$; that is,
  \[
    \lambda^*(z; v) = (z/v)^{1/(N-1)}.
  \] 
  We note that it is a strictly increasing and concave function of $z$, and strictly concave if $N>2$.
  %
  If $N=2$, we have $\lambda^*(z)=z/v$.
  
  Finally, we note here that the logarithmic derivative is 
  \[
    \frac{d\log \lambda^*}{dz}(z)= ((N-1)vz)^{-1}.
  \]

\end{example}


\begin{example}[Second price common value]

  We return to the case of the second price auction with common value $v$.
  %
  Recall that $\lambda^*(z)=(z/v)^{1/(N-1)}$ on $[0,v]$.
  %
  Then 
  \begin{align*}
    U_\Adversary(z) &= v_\Adversary (z/v)^{\frac{N}{N-1}} - Nz(z/v)^{\frac{1}{N-1}} \\
    &= z^{\frac{N}{N-1}}v^{-\frac{1}{N-1}}\cdot \left(v_\Adversary/v - N  \right)
  \end{align*}
  for $z\in[0,v]$.
  %
  This utility function is concave (resp.~convex) if and only if it is negative (resp.~positive) for $z>0$.
  %
  The auction is censorship resistant if and only if $v_\Adversary < Nv$ (concave).

  In the case of a procurement auction with max fee $\pi$ and common marginal cost $\mu$ \cite[\S3.2]{roughgarden2024transaction}, the bound becomes $N(\pi-\mu)$.

\end{example}
\end{comment}


\subsection{Censorship by replacement}
\label{section:replacement}

The approach of censorship by contracts for exclusion is `out of band' in the sense that is makes use of a type of agreement not provided by the typical core blockchain API, whose main purpose is to arrange for transaction \emph{inclusion}.
%
There are also approaches to censorship that use only inclusion interfaces, which we refer to collectively as \emph{censorship by replacement}.

The simplest replacement strategy is to spam transactions offering higher fees than the target until all available blockspace is filled up.
%
In terms of incentives, the spam presents schedulers with a strategy --- to wit, include all of the spam --- that is mutually exclusive with including the targeted flow.
%
However, this approach cannot exclude targeted flow without also excluding \emph{all} transactions with maximum fee at most as much as the target; this approach is therefore necessarily more expensive than out of band contracts for exclusion.

Another example of a censorship strategy which excludes a possibly unnecessarily large class of transactions is so-called `application layer censorship \cite[Def.~4]{wahrstatter2024blockchain}, which attempts to block the state transition that would be induced by a transaction (i.e.~force it to revert).
%
In order for application layer censorship to function at the level of transaction inclusion, the targeted transactions must be attached to \emph{revert protection} guarantees that promise they will not be included into a block if they would revert \cite{zhu2024quantifying}.
%
As with the spam strategy, application layer censorship is generally more costly than a contract for exclusion because it excludes all (protected) transactions that would revert under the manipulated state, and not necessarily only the targeted items.

\subsection{Censorship and efficiency}

Censorship by contracts for exclusion are also distinguished from other approaches to censorship by the fact that it results in an economically inefficient allocation of blockspace --- namely, one in which free blockspace fails to be allocated even though there are available transaction items that offer a fee in excess of the processing costs of some scheduler that could handle it.
%
By way of contrast, an outcome in which blocks are filled with spam that pays higher transaction fees than targeted transaction material, or one that includes higher paying transactions that manipulate state so as to force reversion of targeted transactions, is efficient.
%
From the perspective of the blockspace supply mechanism, such outcomes are a case of the system functioning as intended.

Meanwhile, attacks that focus on manipulating some other layer (such as the network) proceed by modifying the \emph{environment}, for example by limiting the set of transactions visible to each scheduler.
%
They have no bearing on the efficiency of the method by which transactions are selected from that environment for inclusion.

The efficiency argument also provides some microeconomic intuition as to why we might expect introducing competition at the scheduler level might make this particular approach to censorship more costly: in an idealised perfectly competitive scheduler market, there are always suppliers that stand ready to take the place of any censoring scheduler and handle targeted items.
%
In such a perfectly competitive market, censorship by contracts for exclusion cannot occur.

On the other hand, the cost of censorship by replacement strategies remains unchanged in the presence of scheduler competition.

\subsection{Censorship as an anticommons good}

Censorship by contracts for exclusion are often expensive when there are many active schedulers because the adversary faces a tragedy of the anticommons \cite{heller1998tragedy}: if each scheduler may unilaterally decide to accept the item, then in order to get the desired effect, the adversary must make arrangements with \emph{all} schedulers, effectively giving \emph{each} scheduler monopoly power in the deal.

The formulas we obtained in \S\ref{section:model} for equilibrium scheduler strategies assumed that schedulers do not exercise their market power against the adversary (though of course they may against the fee payer).
%
That is, the scheduler accepts an offer whenever it exceeds his opportunity cost, instead of trying to hold out for as much as the censor can afford.
%
Given the reasoning in \S\ref{section:cr} which characterises censorship resistance in terms of the \emph{lowest} returns to censoring under which a censorship strategy is viable, this simplification is quite reasonable.


\subsection{Conditional contracting}

Even among contracts for exclusion, the form of contract we have considered in our model is quite restricted: they make an ex-ante commitment to payment that is not conditioned on any factors other than exclusion.
%
By committing to pay only when certain additional conditions are satisfied, the adversary can modulate their cost of entering into the contracts.

We identify two major types of conditions that may warrant further research:
%
\begin{itemize}
  \item \emph{atomic coalition}. 
    Condition the terms of a contract for exclusion on all schedulers joining the coalition. 
    %
    This effectively makes the construction of a coalition \emph{atomic}.
    %
    Such a contract need only compensate each scheduler for the expected payoffs $\Expectation[\Util_{i,\Population}(\tilde\theta_i)]$ from the fully competitive mechanism.
    
    In such cases, the cost of censorship can actually \emph{fall} as $N$ increases (if $1/\Util_\Population$ grows faster than $N$).
    %
    An adversary who finds a way to coordinate such a contract may therefore have a strong incentive to do so.


  \item \emph{ex-post}.
    Condition the payment on the scheduler winning the auction.
    %
    In this case, only the winning scheduler(s) must be paid, and the compensation must exceed their ex-ante expected profits \emph{conditioned on winning}.

    For example, in the case of a PBS auction \cite{thegostep2021mevboost} such a contract could be implemented by depositing funds in a smart contract and making them available to any entity who can prove that they built the target block and that the target block did not contain the targeted items.
    %
    In this case, the offer must be at least as great as the revenue each builder expects to get from $T$ if he wins, since he can then afford to bid away such revenue to the proposer in order to ensure the win.

\end{itemize}






\subsection{Information asymmetry}
\label{section:asymmetry}


Perhaps the least convincing assumption in our model is that the offer $z$ is common knowledge at the time schedulers choose their strategies.
%
Let us attempt to defend it.

From the perspective of a scheduler, the fact of having received an offer of a contract for exclusion is in itself informative.
%
It strongly suggests that an attempt to censor the targeted flow is ongoing, and hence that the adversary must likely have offered such contracts to all schedulers.
%
Indeed, other than as an attempt at misdirection, I can't think of any other reason to make such offers.
%
Thus even if we do not assume that $i\in\Population$ has knowledge of $z_{-i}$, he can certainly make the inference that $z_i>0 \Rightarrow z_{-i}>0$.

The scheduler might even wish to infer that the censor takes a similar approach to pricing his bribe for all schedulers.
%
For example, if $z_i$ is high enough to make bidder $i$ consider accepting with high probability, then $z_j$ is likely similarly high for all $j\in N$.
%
Implicit in this inference is an assumption that the censor will adopt an approximately \emph{symmetric} strategy, that is, one that targets the same acceptance probability $p^{1/N}$ from all schedulers.
%
However, this assumption cannot generally be justified: even under the common knowledge model, a symmetric offer may not be the adversary's best strategy.

\begin{example}[Asymmetric offer]
  \label{asymmetric-offer}

  We describe here a scenario in which an asymmetric offer achieves a lower cost of censorship at some quantile $p<1$ than a symmetric one.
  %
  Suppose $\Mechanism$ is a second price auction with reserve $\tau$ between two schedulers $a$ and $b$.
  %
  Ties are broken uniformly at random.
  %
  Suppose that the types of $a$ and $b$ are i.i.d.~with distribution function
  %
  \[
    F(t) = \left\{\begin{array}{ll}
      1-q & t<\tau \\
      1 & t \geq\tau
    \end{array}\right.
  \]
  %
  In other words, $\tilde\theta_a$ is $\tau$ with probability $q$ and $0$ otherwise, and similarly for $\tilde\theta_b$, and these events are independent.
  %
  We have
  %
  \[
    \Util_0(0) = \tau \qquad \Util_1(0) = q\tau  \qquad \Util_0(\tau) = \Util_1(\tau) = 0.
  \]
  %

  With probability $q^2$, both players have high costs and hence will accept any nonzero offer to abstain.
  %
  In order to achieve a censorship probability of greater than $q^2$, the offer must be acceptable to both players even when at least one has low costs.
  %
  A player with low costs will never accept an offer of less than $q\tau$ (regardless of the tie breaking rule), which brings the cost of censorship with a symmetric offer to at least $2q\tau$.

  On the other hand, if scheduler $a$ is offered an infinitesimal amount (call it $0$), then he accepts if and only if his costs are high, that is, with probability $q$.
  %
  In this case, if scheduler $b$ handles the item, he receives a profit of $\tau$ if $a$ has high costs and $0$ otherwise.
  %
  That is, the offer of $z_a$ to $a$ has not affected the expected payoffs of $b$, which remain $q\tau$ if his costs are low and $0$ otherwise, whence $b$ will accept any offer in excess of $q\tau$.

  Thus by offering $(0,q\tau)$, the adversary achieves a total censorship probability of $q>q^2$ with a total offer of $q\tau$, which is always less than the lowest feasible symmetric offer of $2q\tau$.
  %
  It follows that for any $p\leq q$, $\CC{p}(\Mechanism)\leq q\tau$, but for $q\in(q^2,q]$, the bound cannot be achieved with a symmetric offer.

\end{example}

If the scheduler does not take his offer $z_i$ as evidence of high offers made to his competitors, then he faces a similar decision scenario to the case of a contract conditioned on the participation of all schedulers.
%
In either case, the alternative to the offer is facing a fully competitive scheduling mechanism, and so the required offer amount may actually end up decreasing as $N$ grows.
%
This information model may be realistic for a risk-averse scheduler $i$, and indeed it is intuitively reasonable that such $i$ may prefer the certainty of the CFE to the uncertainty of the order flow competition.


%%%%%%%%%%%%%%%%%%%%%%%%%%%%%%%%%%%%%%%%%%%%%%%%%%%%%%%%%%%%%%%%%%%%%%%%%%%%%%



\subsection{Sanctions}
\label{section:sanctions}
%
Probably the most widely discussed example of a censorship threat for the Ethereum chain was that presented by the addition of the Tornado Cash contracts to the OFAC sanctions list \cite{wright2022defi}.
%
A regulatory threat such as this can be modelled along the lines of \S\ref{section:model} by positing that a scheduler incurs a certain `risk cost' for breaching the regulation.
%
Such exposures have the following additional characteristics:
\begin{itemize}
  \item They asymmetrically affect agents in different jurisdictions;
  \item Schedulers cannot easily choose whether or not to join the arrangement;
  \item They are long-lived, that is they extend over the period of many blocks;
  \item They tend to be publicly known.
\end{itemize}
%
The long life of such an arrangement means that acceptance probabilities can be understood in a frequentist interpretation as the abstention \emph{rate} of exposed schedulers.

For a simplistic model, suppose given a scheduler population $\Population$ each member of which has a `weight' $\lambda_i\in[0,1]$, such that $\sum_{i\in\Population}\lambda_i=1$ (for example, this could be the stake weight of a consensus node in POS Ethereum).
%
Each epoch, $K$ schedulers are drawn from $\Population$ according to the probability distribution $\vec{\lambda}$ and may participate in a block construction mechanism.
%
To avoid consideration of complicated hypergeometric distributions, we assume the draws are made with replacement, so each scheduler can be selected multiple times.

Now, introduce a censoring adversary who enters into an arrangement --- in the present context, in the form of a regulatory imposition --- with a fraction $\Population_\lambda$ of the population having total weight $\lambda$.
%
Each epoch, if a scheduler $i\in\Population_\lambda$ with type $\tilde\theta_i$ is drawn he faces a competition against up to $K-1$ schedulers of whom an expected fraction $\lambda^{K-1}$ also belongs to the coalition.

Suppose that each scheduler $i$ learns the number $K_i$ of times he has been drawn in a given epoch, and must decide whether or not to abstain before learning whether the other schedulers drawn in that epoch belong to the coalition.
%
In the special case that $i$ knows that his competitors in $\Population_\lambda$ intend to comply with the regulation, he faces the decision inequality
%
\[
  z_i \underset{?}{>} \sum_{k=K_i}^K {K-K_i \choose k-K_i} \left( \frac{\lambda-\lambda_i}{1-\lambda_i} \right)^{K-k} \left(1-\left( \frac{\lambda-\lambda_i}{1-\lambda_i} \right)\right)^{k-K_i} \Util_{i,k-K_i}(\theta_i)
\]
%
where $\Util_{i,k}(\theta_i)$ is the value function of $i$ conditioned on an active set consisting of $k-K_i$ draws with replacement from $\Population\setminus\Population_\lambda$ and $z_i$ is $i$'s estimated risk cost of non-compliance.

The observant reader will note that expression on the right is (a variant of) $Z_i\left(\theta_i, \frac{\lambda-\lambda_i}{1-\lambda_i} \right)$, so that once again we have obtained an expression for the cost of censorship in terms of polynomials whose coefficients are conditional value functions of the underlying mechanism.



\begin{comment}

  \subsection{Negotiation}
\begin{remark}[Negotiation]

  For simplicity, in this paper we mostly assume that the type distributions of schedulers are common knowledge, and that moreover before contracting there is no negotiation phase through which the adversary could elicit additional private information about the type of his counterparty: he must simply make a take-it-or-leave-it offer based on the common prior.

\end{remark}

\end{comment}








\printbibliography
\end{document}

























\begin{comment}
\newpage
\section{Dynamic scheduler population}


\paragraph{Dynamic bidder populations: implementation notes}
In applications, the set of entities authorised to allocate blockspace is listed in some kind of registry.
%
For example:
\begin{itemize}
  \item 
    The set of entities that may propose blocks in Ethereum conensus is indexed by a set 20-byte Ethereum addresses listed in a stake registry. 
    %
    This registry may be updated with the additional or removal of entities.

  \item
    Block builders in PBS must register with a relay to participate in the auction.

  \item
    Preconfirmation providers or `preconfers' in various models of preconfirmations (a type of blockspace forward in which a specific transaction or transaction bundle is guaranteed to land in a certain block) must insert themselves into a registry and put up collateral for their commitments.
\end{itemize}
%
The implementation of the registry is not important for our purposes; as well as a centralised (in the case of the PBS relays) or on-chain (in the case of proposers or preconfers) database, blockspace allocators could simply be the holders of a local authorisation token.

\begin{itemize}
  \item 
    Limits on implementation in smart contracts mean that such mechanisms should have simple rules (so no optimal stopping based on statistical models, yes clock auctions).
  
  \item
    Increasing the number of schedulers can result in increasing computational \cite{buterin2017parametrizing} or consensus \cite{wang2023security} difficulties.
  
  \item
    Certain types of computational tradeoffs are well-studied in the mechanism design literature [cite something about clock auctions?]
    
  \item
    From another perspective, dynamic auctions allow us to posit \emph{dynamic populations} that can expand and contract in response to new information in the market --- for example, information about order flow that has remained unclaimed for some time and hence may be the target of a censorship attempt.
\end{itemize}




\begin{itemize}
  \item
    A scheduler competition where different schedulers arrive over time facilitates the spread of information about other bidders' preferences, and hence in particular, whether censorship attempts may be occurring.

  \item
    So even in an environment where schedulers cannot easily infer the general strategy of an adversary purely from a bribe offered to them, information about that strategy can nonetheless be revealed by the inaction of other schedulers.

  \item
    Sequences of secondary markets for blockspace (resp.~orderflow) should surface price information about that blockspace (orderflow).
    %
    A censorship attempt negatively affects the market price of the target orderflow items.

  \item
    Constructing a grand coalition requires knowledge of the complete scheduler set at the time of contracting (or at the very least, at the earliest time that coalition members could deviate).
    %
    Dynamic scheduler populations undermine this approach; new contracts must be agreed as new schedulers arrive.
    %
    Examples include proposers not in the lookahead, or schedulers taken from a large population eligible to acquire and exercise a transferable blockspace future.

  \item
    There may also be other practical reasons to prefer dynamic scheduler populations, for example messaging complexity.
    %
    A large number of always-active schedulers may have negative implications for throughput or consensus stability \cite{wang2023security}; a small number of always-active schedulers with a larger number sitting in reserve that come into action during a period of market failure could achieve the same efficiency goals without the tradeoffs.

  \item
    For example, in the case of sequentially arriving, short-lived bidders, where contracting with a bidder is not possible before his active period, the product formula bound $\sum_{i=0}^{N-1}(v_i-v^*)$ is realised.
    %
    In this case, it's clearly in the would-be scheduler's interest to remain non-addressable at first, so that the grand coalition contract cannot be conditioned on his behaviour.
    
    Low-cost entry to the scheduler market is key to this result.

\end{itemize}


\begin{itemize}
  \item 
    If we find anything here, it should be that the CR calculations are much simpler and more convincing than in the parallel case.
  \item
    The game sequence where schedulers first are declared active and then enter into contracts to become inactive ($\in\Population_\delta$) again may seem convoluted, but we will argue that this actually most resembles a realistic situation in the presence of markets of secondary holders of transferable blockspace future.

  \item
    In fact, it already somewhat resembles the situation with proposers outside the lookahead, or with miners in a PoW blockchain like Bitcoin.
    %
    These proposers do not directly control their entry into the active set $\Population_\alpha$.

  \item
    For simplicity we assume equal weighted bribes are used, but in reality this approach is probably suboptimal for the asymmetric sequential arrivals model.
    %
    Some cooperative game theory should be used to do it properly.
  \item
    The focus on posted price mechanisms (including with a `fire sale' at the end) is not artificial: it is known that in the cases of interest, these implement efficient allocations.
\end{itemize}

We now anchor ourselves in a dynamic setting in which would-be schedulers must first declare themselves as \emph{active}, incurring a cost $\kappa>0$, before they may participate in the blockspace supply mechanism --- whether by bidding to schedule items or by entering into contracts to abstain.
%
The scheduler population $\Population$ may be infinite, but only finitely many schedulers may declare themselves active at any one time.
%
Denote by $\Population_\alpha$ the active population.

Each would-be scheduler decides whether to declare themselves active, given information available at time $t$, if their expected surplus from participation exceeds the registration cost, i.e.~if
\[
  c_t(v) > \kappa.
\]
We suppose it becomes common knowledge when an item has been scheduled.

\begin{example}[Sequential arrival of short-lived schedulers]

  Suppose that exactly one scheduler may declare itself active at each epoch.
  %
  At the end of the epoch, the active scheduler becomes permanently inactive, i.e.~enters $\Population_\delta$.
  %
  Thus schedulers do not need to consider the value of withholding blockspace for allocation at a later epoch.
  
  Items are advertised with a posted fee $v^*(t)$ (which may depend on the epoch).
  %
  In each epoch, if a scheduler $a$ is active and $v_a<v^*(t)$, the item is allocated to $a$, who receives the fee $v^*$ and retains surplus $v^*-v_a$.
  %
  The (deterministic) surplus for the scheduler that may enter in epoch $t$ is
  \[
    c_t(v) = v^*(t)-v_a
  \]
  if $v^*(t)<v_a$ and $0$ otherwise.
  %
  Entry is feasible iff $v^*(t)-v_a > \kappa$.

  The censor's only strategy is to contract directly with the scheduler in epoch $t$ when it appears at the start of each epoch, at which point the price to beat is $v^*(t)-\tilde{v}_a$.
  %
  To achieve a probability $p>0$ of successful censorship over $N$ epochs, the adversary must pay
  \[
    z(t) = v^*(t) - F_{\tilde{v}}^{-1}(p^{1/N})
  \]
  to the scheduler in epoch $t$, where $F_{\tilde{v}}^{-1}$ is the quantile function of $\tilde{v}$.
  %
  If the posted price $v^*$ is constant, the total cost simplifies to
  \[
    CR = N(v^*-F_{\tilde{v}}^{-1}(p^{1/N})).
  \]
  
\end{example}

\begin{example}[Sequential arrival of short-lived schedulers, static population]

  Suppose allocation works as in the above example, except that the population of schedulers and their assignment to epochs is common knowledge at the start of play.
  %
  Then the $n$th scheduler has an expected surplus of
  %
  \begin{align*}
    c_n(v) &= \Prob[\tilde{v}_k < v^*, k=0,\ldots,n-1] \cdot (v^* - v) \\
    &= F_{\tilde{v}}(v^*)^{n-1}\cdot (v^* - v).
  \end{align*}
  %
  The total cost to subsidise a grand coalition with probability $p$ of success (with symmetric payouts) is
  \begin{align*}
    CR &= \sum_{n=0}^{N-1} F_{\tilde{v}}(v^*)^{n}\cdot (v^* - F_{\tilde{v}}^{-1}(p^{1/N})) \\
    &= \frac{1- F_{\tilde{v}}(v^*)^N}{1-F_{\tilde{v}}(v^*)} (v^* - F_{\tilde{v}}^{-1}(p^{1/N})).
  \end{align*}
  %
  This number is bounded above by $1/(1-F_{\tilde{v}}(v^*))$ as $N\rightarrow\infty$.
  %
  For example, if $v^*$ is chosen at the median of the type distribution so that $F(v^*)=1/2$, the maximum censorship resistance factor is just $2$.

  Now suppose the arriving schedulers know their assignment to epochs in advance, and may \emph{choose} whether or not to make themselves publicly known at the start of the period.
  %
  If $v_\Adversary > N(v^* - v(p^{1/N}))$, then the scheduler can arrange a larger bribe for himself by choosing not to reveal himself until the epoch in which he becomes active.
  %
  On the other hand, if the adversary's budget would cover a grand coalition on a static population declared at the start but not the full cost of a dynamic approach, the scheduler at the back of the queue may be better off declaring himself in advance.

\end{example}

\begin{example}[Sequential arrival of long-lived schedulers]

  If bidders are long-lived, they may have multiple opportunities to bid for an item.
  %
  Conversely, prospective new entrants must consider whether they will face competition from incumbent schedulers who haven't yet received an allocation.

  If the item is not scheduled in a given epoch $t$, it may be inferred that no scheduler with cost $v_i<v^*$ was active during that epoch.
  %
  It follows that $\Population_{\alpha,t} = \Population_{\delta,t}$, and as-yet-inactive members of $\Population$ may estimate their entry surplus $c_{t+1}(v)$ as though they are entering a pristine competition environment.

  Suppose that registration rates are throttled so that at most $N$ new schedulers may register at any time.
  %
  (How are the $N$ new schedulers selected? We haven't allowed them to differentiate based on registration cost $\kappa$.)
  %
  The short answer is that the more schedulers get selected

\end{example}

\end{comment}


\newpage
\appendix

\section{Pointed mechanisms}


\subsection{Pointed mechanisms}

We will study procurement auctions $\mathcal{M}$ with $N$ single parameter bidders and allocation and payment functions 
\[
  x: \uR^N\rightarrow [N] \qquad \pi:\uR^N\rightarrow \uR^n.
\]
%
A key feature is that in addition to any other actions they may take, bidders always have a choice to \emph{abstain} from bidding, regardless of their type.

\begin{definition}

  A \emph{pointed mechanism} is a mechanism $\Mechanism=(\Action,\Theta,U:\Action\times\Theta\rightarrow\R^N)$ each of whose strategy spaces $\Action_i$ are equipped with a distinguished point $\bot\in\Action$ such that
  \[
    U_i(\bot,s_{-i})=0 
  \]
  for all competitor strategy profiles $s_{-i}$ and $s_i$.
  %
  The strategy $\bot$ is called \emph{abstention}.

  If $i\in[N]$, then the \emph{complement} of $i$ in $\mathcal{M}$ is the mechanism $\mathcal{M}_{-i}=(\prod_{j\neq i}\Action_j,\prod_{j\neq i}\Theta_j,V)$ where $V_j$ is defined by the formula
  \[
    V_j(s,\theta_j) = U_j((s,\bot),\theta_j)
  \]
  for each $j\neq i$.
  
  A \emph{compatible sequence} of pointed mechanisms is a sequence $(\mathcal{M}_N)_{N\in\N}$ together with isomorphisms $\mathcal{M}_{N-1}\cong (\mathcal{M}_N)_{-N}$ for each $N$.
  %
  This notion is needed to study large $N$ asymptotics of mechanisms (without having to deal with infinite products).

\end{definition}

Any mechanism can be canonically transformed into a pointed mechanism by formally adding an element $\bot$ to the action space and assigning the utility $U_i(\bot,b_{-i})=0$ for all strategy profiles $b_{-i}\in\Action_{-i}$.

The strategy space $\Action=\prod_i\Action_i$ of a pointed mechanism comes with a function $\Population_\Active(s):\Action\rightarrow 2^\Population$ that picks out the \emph{active set}, that is, the set of participants that did not choose to abstain.

\begin{example}[Direct revelation]

  A direct revelation mechanism has action space equal to the type space $\Theta_i$.
  %
  It can be transformed into a pointed mechanism with strategy space $\Theta_i\sqcup\{\bot\}$.
  %
  Note that some direct revelation mechanisms may already admit a pointing, such as $0\in [0,R]\simeq\Theta_i$.

\end{example}

\begin{definition}[Competitiveness]
  \label{def-competitive}

  A pointed direct revelation mechanism $U:\prod_i\Theta_i\sqcup\{\bot\}\rightarrow\R^\Population$ is \emph{competitive} if the inequality
  %
  \[
    \Util_j(\bot_i,\theta_{-i}) \geq \Util_j(\theta_i,\theta_{-i})
  \]
  %
  for every $\theta_i\in\Theta_i$, $\theta_{-i}\in\Theta_{-i}$.

\end{definition}

Suppose $\Theta_i\subseteq\R$.
%
A (direct revelation) selection mechanism is \emph{efficient} if for all $\theta\in\overline{\Theta}$, $\theta_{x(\theta)}$ is maximal among the active coefficients of $\theta$.
%
Every efficient mechanism is competitive.

\begin{definition}[Single scheduler selection]

  A \emph{single scheduler selection mechanism} is a tuple $\Mechanism=(\Population, \tilde\theta, \Action=\prod_{i\in\Population}\Action_i,x:\Action\rightarrow\Population,f:\Action\rightarrow\uR^\Population)$, where
  \begin{enumerate}
    \item $\Population$ is a finite set;
    \item $F_{\tilde\theta}$ is a \emph{cost} distribution on $\uR^\Population$;
    \item $\Action_i$ is a pointed set;
    \item $x(s) \neq i$ if $s_i=\bot$;
    \item $f(s) = 0$ if $s_i=\bot$.
  \end{enumerate}
  %
  We also assume that the total fee payment is bounded above by some \emph{max fee} $\tau>0$, i.e.~$\sum_if_i(s)\leq\tau$.
  
  We associate a pointed mechanism to a single scheduler selection mechanism by setting $\Theta_i=\uR$ and
  \[
    U_i(s,\theta_i) = f(s) - I_{x(s)=i}\theta_i.
  \]
  %
  If $j\in \Population$, we can define the \emph{complement of $j$} as the selection mechanism $(\Population\setminus\{j\},\tilde\theta,\prod_{i\neq j}\Action_i,x',f')$ where $x'(s_{-j}) = x(\bot,s_{-j})$ and $f(s_{-j})=f(\bot,s_{-j})$; the hypotheses on $x$ and $f$ ensure that these belong to $\Population\setminus\{j\}\subset\Population$ and $\Action_{-j}\simeq \Action_{-j}\times\{\bot\}\subset \Action$, respectively.

  A necessary condition for competitiveness of the associated pointed mechanism is that there is no strategy profile $s$ such that $f_i(s)>0$ and $x(s)=0$.
  %
  The competitiveness condition is
  %
  \[
    \Prob(x(s)=i \mid s_j = \bot) \geq \Prob(x(s)=i \mid s_j\neq\bot)
  \]
  %
  for all $i\neq j$.

\end{definition}


\subsection{Conditional values}

\begin{definition}

  The \emph{conditional value} of player $i$ with active set $S\subseteq \Population$ is
  \[
    \Util_{i,S}(\theta) = \Expectation[\Util_i(\tilde{\theta}) \mid \Population_\alpha(s)=S].
  \]
  %
  In the case of a single scheduler selection mechanism $x:\Action\rightarrow\Population$, $f:\Action\rightarrow\uR^\Population$ with quasi-linear payoffs, by Myerson's Lemma we have
  %
  \[
    \Util_{i,S}(\theta) = \Util_{i,S}(\theta') + \int_{\theta}^{\theta'} \Prob[i \text{ wins} \mid \tilde{\theta}_i = s,\,\Population_\alpha = S] ds.
  \]
  %
  Suppose that the item offers a maximum fee $\tau$, so that $\sum_if_i(\vec{s}) \leq\tau$.
  %
  Then $\Util_{i,S}(\theta) \leq [\tau - \theta]_+$ in all cases, and so
  %
  \[
    \Util_{i,S}(\theta) = \int_{\theta}^\tau \Prob[i \text{ wins} \mid \tilde{\theta}_i = s,\,\Population_\alpha = S] ds.
  \]


  If the mechanism is symmetric, then this function is independent of $i$ and depends on $S$ only through its cardinality; we write
  \[
    \Util_{n}(v) = \Expectation[\Util_i(v) \mid \Population_\alpha(\theta_{-i})=S]
  \]
  for $n\in\N$ and any $S$ with cardinality $n$.
  
\end{definition}


\begin{example}[Vickrey auction with uniform type distribution]

  Suppose $\Mechanism$ is a second price auction with bidder types uniformly distributed on $[0,1]$.
  %
  Then 
  \[
    \Util_k(\theta) = \theta \left( 1 - \frac{k\theta^k}{k+1} \right)
  \]
  for all $k$ and $\theta\in[0,1]$.
  
  One can show that the coefficients of $Z(p,1-p)(1)$ are all positive.

\end{example}

\begin{lemma}[Monotonicity of conditional values]
  \label{conditional-value-monotonic}

  Suppose that $\Mechanism$ is competitive.
  %
  Then for any $i,j\in\Population$ with $i\neq j$, $S\subseteq\Population\setminus\{i,j\}$, and $\theta_i\in\Theta_i$, the inequality
  \[
    \Util_{i,S}(\theta_i) \geq \Util_{i,S\sqcup\{j\}}(\theta_i)
  \]
  %
  is satisfied.

\end{lemma}
%
\begin{proof}

  Immediate from the definition \ref{def-competitive} of competitiveness. \qedhere

\end{proof}

\begin{proposition}[Surplus equations for efficient mechanisms]

  If $\Mechanism$ is an efficient auction with max fee $\tau$, then
  \[
    \frac{\partial \Util_k}{\partial\theta} ( \theta ) = -F_{\tilde{\theta}}(\theta)^k
  \]
  for all $k\geq 0$, where $F_{\tilde{\theta}}$ is the CDF of the cost distribution.

\end{proposition}

\begin{example}[Single item auction]

  In the focus case of a single item auction (where the item cannot be awarded to any abstaining bidder), we have
  \[
    c_{i,S}(v) = \Prob[x(b)=i\mid \Active(b_{-i})=S]\cdot v - \Expectation[\pi_i(b)\mid \Active(b_{-i})=S].
  \]
  A procurement auction is handled the same way, except the valuation $v<0$ represents the \emph{cost} of providing the service and payments $\pi_i<0$ go the other way.

\end{example}

It's worth singling out the coefficient $c_0(v)=c_\emptyset(v)$ as the \emph{monopolist} case.
%
In an efficient single item procurement auction, this number is $f_{\max}-v$ where $f_{\max}$ is the auction \emph{reserve price}, that is, the maximum fee the auctioneer is prepared to pay.

The \emph{gap} $c_0-c_1$ may be substantial.

Moreover, by revenue equivalence we have $c_0 \geq c_1 \geq \cdots \geq c_{N-1}$ for any mechanism that implements the efficient allocation, with all inequalities strict if the bidder type distribution is atom free.



\begin{lemma}

  For any profile of optimal strategic plans, $\Util_i(\theta_i,z_{-i})$ is monotone decreasing in all arguments.

\end{lemma}
%
\begin{proof}

  By Myerson's Lemma \cite[Thm.~3.3]{milgrom2004putting}, each coefficient $c_{i,S}(\theta_i)$ is monotone decreasing in $\theta_i$, while the monomials $\bar\lambda_j(z_j)$ are independent of $\theta_i$.
  %
  Meanwhile, $c_{i,S}(\theta_i)$ is independent of $z_j$ by construction, while $\bar\lambda_j(\vec{z})$ is increasing in $z_j$.
  %
  Write
  \begin{align*}
    c_i(\theta_i,z_{-i}) =& \bar\lambda_j(z_j) \sum_{S\ni j} c_{i,S}(\theta_i) \prod_{r\in S\setminus j}\bar\lambda_r(z_r) \prod_{i\neq k\not\in S} (1-\bar\lambda_k(z_k)) \\ 
    &+ (1-\bar\lambda_j(z_j))\sum_{S\not\ni j} c_{i,S}(\theta_i) \prod_{r\in S}\bar\lambda_r(z_r) \prod_{k\in\Population\setminus(S\sqcup\{i,j\})} (1-\bar\lambda_k(z_k)).
  \end{align*}
  %
  Increasing $\bar\lambda_j(z_j)$ moves weight from the second term to the first term, so the result will follow from the inequality
  %
  \[
    \sum_{S\ni j} c_{i,S}(\theta_i) \prod_{r\in S\setminus j}\bar\lambda_r(z_r) \prod_{k\in\Population\setminus\{i\}} (1-\bar\lambda_k(z_k)) \\ 
    <\sum_{S\not\ni j} c_{i,S}(\theta_i) \prod_{r\in S}\bar\lambda_r(z_r) \prod_{k\in\Population\setminus(S\sqcup\{i,j\})} (1-\bar\lambda_k(z_k))
  \]
  %
  which can be shown by summing the inequality
  %
  \[
    c_{i,S}(\theta_i) \geq c_{i,S\sqcup\{j\}}(\theta_i)
  \]
  %
  (Lemma \ref{conditional-value-monotonic}) over $S\subset \Population\setminus\{i,j\}$. \qedhere

\end{proof}



\subsection{Payoff generating function}


These assemble into a \emph{payoff generating function}
\[
  Z(p,q;v) = \sum_{k=0}^N c_{k}(v){N\choose k} q^kp^{N-k}.
\]
For fixed $v$, $Z(p,q;v)$ is a polynomial in $p$ and $q$.

\begin{lemma} $\frac{\partial Z_N}{\partial p} = NZ_{N-1}$ \end{lemma}

\begin{lemma}[Positivity of coefficients]

  The coefficients of $\frac{\partial}{\partial p}Z(p,1-p)$ are all non-negative.
  %
  If $\mathcal{M}$ is gapped, then at least one coefficient is positive.
  %
  The same goes for $\frac{\partial^2}{\partial p^2}Z(p,1-p)$.
  %
  In particular, $Z(p,1-p)$ is smoothly invertible and convex on $[0,1]$.

\end{lemma}
%
\begin{proof}

  It's enough to show that the coefficients of $(\partial/\partial p - \partial/\partial q)Z(p,q)$ are all positive.
  %
  We have
  \begin{align*}
    \frac{\partial Z}{\partial p}(p,q) &= \sum_{k=0}^{N-1} c_k {N \choose k}(N-k)p^{N-k-1}q^k \\
    \frac{\partial Z}{\partial q}(p,q) &= \sum_{k=1}^N c_k {N \choose k}kp^{N-k}q^{k-1} \\
    &= \sum_{k=0}^{N-1} c_{k+1} {N \choose k+1}(k+1) p^{N-k-1}q^{k}.
  \end{align*}
  So the coefficient of $p^{N-k-1}q^k$ is given by
  \[
    c_k {N\choose k}(N-k) - c_{k+1} {N\choose k+1}  (k+1) =  \frac{N!}{k!(N-k-1)!} (c_k-c_{k+1}) > 0. 
  \]
  It follows similarly that the coefficient of $p^{N-k-2}q^k$ in the second derivative is $\frac{N!}{k!(N-k-2)!}(c_k-c_{k+2})\geq 0$.

\end{proof}

\paragraph{Inverse of $Z(p,1-p)$}

Differentiating, we have
\begin{align*}
  Z'(q) &= -c_0(N-1)(1-q)^{N-2} + c_1(N-1)(1-q)^{N-2} + O(q) \\
  &= (N-1)(1-q)^{N-2}(c_1-c_0) + O(q) \\
  &= (N-1)(c_1-c_0) + O(q)
\end{align*}
which is strictly negative in a neighbourhood of $q=0$; unsurprising, since we expect that lowering the offer increases the probability of participation.
%
Therefore $Z(q)$ has an analytic inverse $q^*=G(z)$ near $q=0$, and 
\[
  G'(Z(q)) = 1/F'(q) \approx \frac{1}{(N-1)(c_1-c_0)}.
\]

\subsection{Random inequalities I haven't been able to use}

\begin{itemize}

  \item If $Q$ is the quantile function of $F$, then $Q(F(z))\leq z$ and $F(Q(p))\geq p$.
  \item Quantile functions are additive over non-negative, monotone increasing functions of a random variable: $Q_{\Util(\tilde\theta)} = \sum_{k=1}^N Q_{\Util_{k-1}(\tilde\theta)} \cdot stuff$.
  \item For an efficient single item auction, $Q_{\Util_0(\tilde\theta)} = \tau + Q_{-\tilde\theta}$.
  \item But the (infimum) quantile function doesn't make sense in the multivariate case.

\end{itemize}


\begin{lemma}

  \label{thm:quantile-sum}

  We have $Q_{\Util(\tilde\theta,\vec{z})} = \sum_{k=1}^N { N-1 \choose k-1 } Q_{\Util_{k-1}(\tilde\theta)} \bar\lambda(\vec{z})^{N-k} (1-\bar\lambda(\vec{z}))^{k-1}$.

\end{lemma}
%
\begin{proof}

  Because each $\Util_k$ is a monotone decreasing function of $\tilde\theta$, each value of $\Util(\tilde\theta,\vec{z})$ is achieved at the same time as at most one set of values of $(\Util_k(\tilde\theta))_{k=0}^{N-1}$. \qedhere

\end{proof}


\begin{lemma}

  Let $\tilde\theta$ be a real valued random variable and let $g:\R\rightarrow\R$ be a monotone decreasing function defined on the support of $\tilde\theta$.
  %
  Then for any $p\in[0,1]$, $g(\hat Q_{\tilde\theta}(1-p)) \geq Q_{g(\tilde\theta)}(p)$, where $\hat Q_{\tilde\theta} = -Q_{-\tilde\theta}$ is the `lower quantile' function of $\tilde\theta$.

\end{lemma}

\printbibliography