\maketitle
%\thispagestyle{fancy}

\begin{abstract}

  We consider the effectiveness of the heuristic $\mathrm{CR} = N\pi$ for evaluating censorship resistance of a blockspace supply chain mechanism.
  %
  In a scenario where a censoring adversary may contract directly with each scheduler to obtain commitments to non-inclusion, we find the bound is pretty good in practice (though hard to compute exactly in realistic models).
  
  If, on the other hand, the adversary can fund an \emph{atomic} coalition --- one in which bribes are only paid out if \emph{all} schedulers join --- then the cost of censorship can actually go \emph{down} with the number of schedulers.
  %
  So the cost of censorship is quite sensitive to the details of the available types of contracts for non-inclusion.

  Essentially the same conclusions hold when introducing multiple schedulers in sequence rather than in parallel, and even in hybrid models where schedulers have distinct, but overlapping, periods of activity.

\end{abstract}

\section{Introduction}

\begin{itemize}

  \item
    The basic market structure considered in this paper is one in which \emph{schedulers}, who have authorisation to commit items to the schedule, provide a scheduling service to order flow managers, who have transaction \emph{items} they wish to get into the schedule.

    Both blockspace and orderflow may have their own upstream reseller markets, but we assume that the purpose of any such markets is essentially to achieve more efficient matching of items with blockspace.

  \item
    We posit an (informal) efficiency objective: to maximise the welfare achievable by matching items to blockspace.
    %
    The welfare of an allocation $\{(\tau_i,L_i)\}_{i\in I}$ that attaches item $\tau_i$ to blockspace lease $L_i$ is $f_{\tau_i}-v_{L_i}$, where $f_i$ is the value to the OFO of getting the item into the schedule, and $v_{L_i}$ is the cost of the blockspace issuer of sequencing the an item into the lease $L_i$.
    %
    (\cite{roughgarden2024transaction} argues that the marginal orphan risk can be used as a principle to estimate $v_i$.)
    
  \item
    If the schedule is not congestible, a welfare maximising outcome assigns all items to the lowest cost blockspace.
    %
    An outcome in which some outstanding item $\tau_i$ with value $f_i$ fails to be scheduled, even though there be blockspace $L_i$ with cost $v_i<f_i$, is inefficient.
    
    Such outcomes could in principle arise in the case that $\tau_i$ was censored by direct contracting.
    %
    The allocation of such contracts (for non-inclusion) is not part of the definition of welfare of our scheduling system, so such contracts may actually decrease welfare even if they are net profitable for all participants.

  \item 
    On the other hand, if the block is congestible, then such allocations can fail in an efficient outcome if congestion occurs.
    %
    In particular, outcomes in which there is deliberate \emph{censorship by congestion} can be efficient.

  \item 
    Unilateral censorship by a scheduler is an expression of market power as control over the allocation. 
    %
    Typically, introducing competition into the scheduler market reduces the market power of each scheduler, and hence their ability to censor, i.e.~force an inefficient outcome.

  \item
    We argue that the effect of the artificial-seeming conditional tip system of \cite{FPR}, or close to it, can in fact be achieved under quite general conditions of market competition among schedulers.
    
  \item
    These models depend on the assumption that the adversary can only achieve censorship by direct contracting with each scheduler individually.
    %
    They break down substantially if we allow the adversary to form a \emph{grand coalition} where bribes are paid out only in the event that all schedulers subscribe to the coalition.
    %
    In these cases, increasing the number of schedulers can actually \emph{decrease} the cost of censorship.

  \item
    However, order can be restored if we allow \emph{dynamic scheduler populations}; in such cases, the calculations actually become quite simple (with some caveats).

\end{itemize}


\paragraph{What is censorship?}
%
In blockchain, censorship is generally characterised as a deliberate attempt to prevent or delay a transaction or class of transactions from being added to the chain.
%
Because \emph{intent} forms part of the definition, censorship can be difficult to identify or measure.
%
It can be achieved in various ways, for example:
\begin{itemize}
  \item Network layer censorship --- eclipse attack, spam
  \item Consensus level Byzantine equivocation
  \item System state manipulation (e.g. governance attack, randomness manipulation)
  \item Blockspace congestion
\end{itemize}

Previous work \cite{FPR} has focused on censorship via \emph{direct contracting}, where an adversary bargains directly with \emph{schedulers}, that is any agents capable of committing transaction items to the chain, for non-inclusion of the censorship target.
%
This type of censorship may seem more game-theoretically tractable --- indeed, in principle the censor must simply compensate each scheduler for whatever revenue he forgoes by omitting the censored items; the cost of such compensation is then understood as quantifying censorship resistance --- but is it actually relevant?
%
Is this approach to censorship plausible, or if not, does it at least bound the costs of other more likely approaches to censorship?
%
Are the figures obtained from these censorship resistance calculations economically meaningful?

Here are the answers to these questions:
\begin{itemize}
  \item
    The first point is a positive one: censorship by direct contracting is distinguished from certain other types of censorship by being \emph{economically inefficient}, where we define economic efficiency carefully as optimising a welfare functional defined in terms of including transaction items into a schedule (but not including other contract predicates).

    Censorship by congestion and by state manipulation, while problematic for users, is part of the normal functioning of the underlying scheduling system --- it is the system working as intended.
    %
    Solutions to these problems, if they exist, are the responsibility of other components.
    %
    For example, censorship by congestion is made more costly by increasing throughput.

  \item
    The second point is equivocal.
    %
    The quantities obtained for under the model of \cite{FPR} and its generalisations considered in this paper are highly sensitive to the particular transaction items under consideration, and are not invariants of the underlying blockspace supply mechanism.
    
    It is nonetheless tempting to adopt the simple intuition of the \emph{product formula}
    \[
      CR = f_\mathrm{max} \cdot N
    \]
    where $f_\mathrm{max}$ is the maximum marginal payoff associated with the targeted orderflow and $N$ is the number of schedulers able to allocate it.
    %
    (Here the `maximum' means maximum over all blockspace allocation mechanisms and scheduler populations.)
    %
    The quantity $f_\mathrm{max}$ is the part that's particularly sensitive to the current state of the mempool and the incentives of schedulers.
    %
    At the moment, I don't know a good way to factor out this number and obtain an invariant of the underlying blockspace supply chain.

    Nonetheless, if we accept this intuition it may be reasonable to conclude that, \emph{holding all other considerations equal}, increasing $N$ results in increased censorship resistance.
    %
    Does this conclusion hold up to scrutiny?

  \item
    It is not difficult to come up with models in which the overly simplistic product formula fails.
    %
    First, under the same censorship contracting environment as in \cite{FPR} and with competitive transaction scheduling being allocated through a second price auction, marginal censorship strategies with a non-negligible (but also not high) probability of success, e.g.~$p=0.1$, become feasible, depending on the distribution of the marginal costs of schedulers.
    %
    The product formula is derived under assuming a `sure success' censorship strategy with complete information. 
    
  \item
    Much more dramatic failures can be obtained if we give the adversary the ability to contract atomically with multiple schedulers at once.
    %
    In such cases, censorship resistance can actually \emph{fall} as $N$ increases.

  \item
    A secondary conclusion is that \emph{increasing $f_\mathrm{max}$} by whatever means are available increases censorship resistance.
    %
    (One possible such means is by bundling diverse orderflow together so that individual transactions cannot be censored without rejecting entire batches.)
    %
    But under competitive scheduling environments, this too may fail; competitive pressures may mean the reserve price is never called upon.

\end{itemize}


\paragraph{Remarks on increasing the number of bidders}

There may be some limits on increasing $N$ arbitrarily; for example:
\begin{itemize}
  \item Communication complexity
  \item Consensus problems (\cite{filecoin})
\end{itemize}
Communication complexity issues can be somewhat mitigated by holding a \emph{dynamic} auction instead of a static one; for example, the descending clock (Dutch) auction, while strategically equivalent to first price sealed bid, requires only one bidder to send a message.
%
From another perspective, dynamic auctions allow us to posit \emph{dynamic populations} that can expand and contract in response to new information in the market --- for example, information about order flow that has remained unclaimed for some time and hence may be the target of a censorship attempt.
%
We should therefore permit dynamic auctions in our framework.

Limits on implementation in smart contracts mean that such mechanisms should have simple rules (so no optimal stopping based on statistical models, yes clock auctions).

Consensus problems are subtler and I will not try to address mitigation strategies here.

\paragraph{Bidder registry}
In applications, the set of entities authorised to allocate blockspace is listed in some kind of registry.
%
For example:
\begin{itemize}
  \item 
    The set of entities that may propose blocks in Ethereum conensus is indexed by a set 20-byte Ethereum addresses listed in a stake registry. 
    %
    This registry may be updated with the additional or removal of entities.

  \item
    Block builders in PBS must register with a relay to participate in the auction.

  \item
    Preconfirmation providers or `preconfers' in various models of preconfirmations (a type of blockspace forward in which a specific transaction or transaction bundle is guaranteed to land in a certain block) must insert themselves into a registry and put up collateral for their commitments.
\end{itemize}
%
The implementation of the registry is not important for our purposes; as well as a centralised (in the case of the PBS relays) or on-chain (in the case of proposers or preconfers) database, blockspace allocators could simply be the holders of a local authorisation token.

A censoring adversary $\Adversary$ must successfully bribe all blockspace managers that could handle the target OF.
%
In a static blockspace auction, if the list of blockspace handlers is publicly accessible, bids are public, and can be matched to an entry in the registry, then this task may be feasible.
%
Indeed, $\Adversary$ could publicly announce his offer of payment for abstention and later validate which handlers in the registry did in fact abstain before completing payments.

The task becomes infeasible if the registry contents isn't known to the adversary.
%
This may be possible with cryptographic techniques.

\subsection{Parallel versus sequential competition}

\begin{itemize}
  \item
    A scheduler competition where different schedulers arrive over time facilitates the spread of information about other bidders' preferences, and hence in particular, whether censorship attempts may be occurring.

  \item
    So even in an environment where schedulers cannot easily infer the general strategy of an adversary purely from a bribe offered to them, information about that strategy can nonetheless be revealed by the inaction of other schedulers.

  \item
    Sequences of secondary markets for blockspace (resp.~orderflow) should surface price information about that blockspace (orderflow).
    %
    A censorship attempt negatively affects the market price of the target orderflow items.

  \item
    Constructing a grand coalition requires knowledge of the complete scheduler set at the time of contracting.
    %
    Dynamic scheduler populations undermine this approach; new contracts must be agreed as new schedulers arrive.
    %
    Examples include proposers not in the lookahead, or schedulers taken from a large population eligible to acquire and exercise a transferable blockspace future.

  \item
    There may also be other practical reasons to prefer dynamic scheduler populations, for example messaging complexity.
    %
    A large number of always-active schedulers may have negative implications for throughput or consensus stability \cite{wang2023security}; a small number of always-active schedulers with a larger number sitting in reserve that come into action during a period of market failure could achieve the same efficiency goals without the tradeoffs.

  \item
    For example, in the case of sequentially arriving, short-lived bidders, where contracting with a bidder is not possible before his active period, the product formula bound $\sum_{i=0}^{N-1}(v_i-v^*)$ is realised.
    %
    In this case, it's clearly in the would-be scheduler's interest to remain non-addressable at first, so that the grand coalition contract cannot be conditioned on his behaviour.
    
    Low-cost entry to the scheduler market is key to this result.

  \item
    A related observation I'm still trying to fit into the general argument: censoring or other `underhanded' coalitions, due to negotiation costs, are generally more likely to be associated with long-term arrangements for repeated censorship of a class of transactions.
    %
    Censorship succeeds when transactions must be sequenced during a period where all active schedulers belong to the censoring coalition.

\end{itemize}






\subsection{Examples}

Some example scenarios I want to write about:
\begin{enumerate}
  \item
    Suppose, as in \cite{FPR}, that receipt of a bribe offer is taken as evidence that a censorship attempt is ongoing, with the assumption that the amount offered to other parties is similar.
    %
    Then a high bribe offer suggests a censorship that is likely to succeed, which also means that if the target agent rejects the bribery offer he is likely to face reduced competition and hence higher profits from the auction.
    %
    For this reason, symmetric bribes generally need to beat the expected \emph{monopoly} profit from the auction to have a high probability of success.

    However, a low but non-negligible probability of success can sometimes be achieved with a \emph{low} bribe.
    %
    Indeed, a bidder that ignores a low bribe is still likely to face some competition in the scheduler auction, so the bribe only needs to beat the \emph{competitive} profits (and monopoly profits with low probability).
    %
    We formalise this line of thinking in Example \ref{ex:non-negligible}.

  \item
    On the other hand, if a bribe is \emph{not} taken indicative of a censorship attempt that affects other bidders too, then more bidders can, apparently paradoxically, entail lower total bribery cost.
    %
    This is because the bribe only needs to beat competitive profits for each of $N$ bidders, which may be less than $1/N$ times the monopoly profits.

    However, I found it difficult to come up with any reasons that this model would be realistic.
    %
    One option is to have a long-lived adversary regularly issue `false flag' censorship bribes to individual bidders.
    %
    If bidders cannot see bribe offers extended to their competitors, then they cannot tell whether a real censorship attempt is underway.

  \item
    Another, perhaps more immediately relevant, way that bribes can arise is in the form of threats, for example from regulators (OFAC) or legal contracts.
    %
    A legally binding commitment or exposure of this type encumbers the affected agent with a \emph{risk cost} $z$ for breaching (for example, an added insurance premium or legal representation cost).
    %
    Such exposures have the following characteristics:
    \begin{itemize}
      \item They asymmetrically affect agents in different jurisdictions; there is no reason to expect all bidders to be affected by one regulatory system.
      \item They are long-lived.
      \item They tend to be public. (In the case of some legal contracts such as non-compete agreements, they may not be public, but this example is rather speculative.)
    \end{itemize}

\end{enumerate}

Then we also have to talk about different mechanism examples.
%
Like block a, block b とか。


%%%%%%%%%%%%%%%%%%%%%%%%%%%%%%%%%%%%%%%%%%%%%%%%%%%%%%%%%%%%%%%%%%%%%%%%%%%%%%%
\newpage
\section{Censorable auctions}

\subsection{Pointed mechanisms}

We will study procurement auctions $\mathcal{M}$ with $N$ single parameter bidders and allocation and payment functions 
\[
  x: \uR^N\rightarrow [N] \qquad \pi:\uR^N\rightarrow \uR^n.
\]
%
A key feature is that in addition to any other actions they may take, bidders always have a choice to \emph{abstain} from bidding, regardless of their type.

\begin{definition}

  A \emph{pointed mechanism} is a mechanism $\mathcal{M}=(\Action,\Theta,U:\Action\times\Theta\rightarrow\R^N)$ each of whose strategy spaces $\Action_i$ are equipped with a distinguished point $\bot\in\Action$ such that
  \[
    U_i(\bot,b_{-i})=0 
  \]
  for all competitor strategy profiles $b_{-i}$.
  %
  The strategy $\bot$ is called \emph{abstention}.

  If $i\in[N]$, then the \emph{complement} of $i$ in $\mathcal{M}$ is the mechanism $\mathcal{M}_{-i}=(\prod_{j\neq i}\Action_j,\prod_{j\neq i}\Theta_j,V)$ where $V_j$ is defined by the formula
  \[
    V_j(s,\theta_j) = U_j((s,\bot),\theta_j)
  \]
  for each $j\neq i$.
  
  A \emph{compatible sequence} of pointed mechanisms is a sequence $(\mathcal{M}_N)_{N\in\N}$ together with isomorphisms $\mathcal{M}_{N-1}\cong (\mathcal{M}_N)_{-N}$ for each $N$.
  %
  This notion is needed to study large $N$ asymptotics of mechanisms (without having to deal with infinite products).

\end{definition}

Any mechanism can be canonically transformed into a pointed mechanism by formally adding an element $\bot$ to the action space and assigning the utility $U_i(\bot,b_{-i})=0$ for all strategy profiles $b_{-i}\in\Action_{-i}$.

The strategy space $\Action=\prod_i\Action_i$ of a pointed mechanism comes with a function $\Active:\Action\rightarrow[N]$ that picks out the \emph{active set}, that is, the set of participants that did not choose to abstain.

\begin{example}[Direct revelation]

  A direct revelation mechanism has action space equal to the type space $\Theta_i$.
  %
  It can be transformed into a pointed mechanism with (pure) strategy space $\Theta_i\sqcup\{\bot\}$.
  %
  Note that some direct revelation mechanisms may already admit a pointing, such as $0\in [0,R]\simeq\Theta_i$.

\end{example}


\begin{definition}[Surplus coefficients]

  The \emph{expected surplus} of player $i$ with active set $S\subseteq \Population$ is
  \[
    c_{i,S}(v) = \Expectation[U_i(v) \mid \Population_\alpha(b_{-i})=S].
  \]

  If the mechanism is symmetric, then this function is independent of $i$ and depends on $S$ only through its cardinality; we write
  \[
    c_{n}(v) = \Expectation[U_i(v) \mid \Population_\alpha(b_{-i})=S]
  \]
  for $n\in\N$ and any $S$ with cardinality $n$.
  %
  These assemble into a \emph{surplus generating function}
  \[
    Z(p,q;v) = \sum_{k=0}^N c_{k}(v){N\choose k} q^kp^{N-k}.
  \]
  For fixed $v$, $Z(p,q;v)$ is a polynomial in $p$ and $q$.
  
\end{definition}

\begin{example}[Single item auction]

  In the focus case of a single item auction (where the item cannot be awarded to any abstaining bidder), we have
  \[
    c_{i,S}(v) = \Prob[x(b)=i\mid \Active(b_{-i})=S]\cdot v - \Expectation[\pi_i(b)\mid \Active(b_{-i})=S].
  \]
  A procurement auction is handled the same way, except the valuation $v<0$ represents the \emph{cost} of providing the service and payments $\pi_i<0$ go the other way.

\end{example}

It's worth singling out the coefficient $c_0(v)=c_\emptyset(v)$ as the \emph{monopolist} case.
%
In an efficient single item procurement auction, this number is $f_{\max}-v$ where $f_{\max}$ is the auction \emph{reserve price}, that is, the maximum fee the auctioneer is prepared to pay.

The \emph{gap} $c_0-c_1$ may be substantial.

Moreover, by revenue equivalence we have $c_0 \geq c_1 \geq \cdots \geq c_{N-1}$ for any mechanism that implements the efficient allocation, with all inequalities strict if the bidder type distribution is atom free.


\begin{definition}

  The \emph{Euler characteristic} of a mechanism $\mathcal{M}_N$ with $N$ players is
  \[
    \chi(\mathcal{M}_N)(v) := \sum_{k=0}^{N} (-1)^k { N \choose k } c_k(v).
  \]
  For the FPR mechanism, this number is $T-t$ as long as $v\leq t$.

\end{definition}

\begin{lemma} $\frac{\partial Z_N}{\partial p} = NZ_{N-1}$ \end{lemma}

\begin{lemma}[Positivity of coefficients]

  The coefficients of $\frac{\partial}{\partial p}Z(p,1-p)$ are all non-negative.
  %
  If $\mathcal{M}$ is gapped, then at least one coefficient is positive.
  %
  The same goes for $\frac{\partial^2}{\partial p^2}Z(p,1-p)$.
  %
  In particular, $Z(p,1-p)$ is smoothly invertible and convex on $[0,1]$.

\end{lemma}
%
\begin{proof}

  It's enough to show that the coefficients of $(\partial/\partial p - \partial/\partial q)Z(p,q)$ are all positive.
  %
  We have
  \begin{align*}
    \frac{\partial Z}{\partial p}(p,q) &= \sum_{k=0}^{N-1} c_k {N \choose k}(N-k)p^{N-k-1}q^k \\
    \frac{\partial Z}{\partial q}(p,q) &= \sum_{k=1}^N c_k {N \choose k}kp^{N-k}q^{k-1} \\
    &= \sum_{k=0}^{N-1} c_{k+1} {N \choose k+1}(k+1) p^{N-k-1}q^{k}.
  \end{align*}
  So the coefficient of $p^{N-k-1}q^k$ is given by
  \[
    c_k {N\choose k}(N-k) - c_{k+1} {N\choose k+1}  (k+1) =  \frac{N!}{k!(N-k-1)!} (c_k-c_{k+1}) > 0. 
  \]
  It follows similarly that the coefficient of $p^{N-k-2}q^k$ in the second derivative is $\frac{N!}{k!(N-k-2)!}(c_k-c_{k+2})\geq 0$.

\end{proof}

\begin{example}[Vickrey auction with uniform type distribution]

  Suppose $\mathcal{M}$ is a second price auction with bidder types uniformly distributed on $[0,1]$.
  %
  Then 
  \[
    c_k(v) = v \left( 1 - \frac{kv^k}{k+1} \right)
  \]
  for all $k$ and $v\in[0,1]$.
  
  One can show that the coefficients of $Z(p,1-p)(1)$ are all positive.

\end{example}

\paragraph{Inverse of $Z(p,1-p)$}

Differentiating, we have
\begin{align*}
  Z'(q) &= -c_0(N-1)(1-q)^{N-2} + c_1(N-1)(1-q)^{N-2} + O(q) \\
  &= (N-1)(1-q)^{N-2}(c_1-c_0) + O(q) \\
  &= (N-1)(c_1-c_0) + O(q)
\end{align*}
which is strictly negative in a neighbourhood of $q=0$; unsurprising, since we expect that lowering the offer increases the probability of participation.
%
Therefore $Z(q)$ has an analytic inverse $q^*=G(z)$ near $q=0$, and 
\[
  G'(Z(q)) = 1/F'(q) \approx \frac{1}{(N-1)(c_1-c_0)}.
\]

\newpage
\subsection{Bidder decision problem}

Suppose that after bidder $i$ draws his type, but before other actions are taken, an adversary $\Adversary$ offers $i$ a bribe $z_i$ to abstain from the auction.
%
Bidder $i$ has two pure strategies at this stage:
\begin{enumerate}
  \item Reject the bribe and play his optimal strategy in the rest of the game.
  \item Accept the bribe and abstain from play, netting $z_i$.
\end{enumerate}
%
More generally, $i$ may choose an arbitrary mixture $\lambda_i(v_i,z_i)\in[0,1]$ of the two strategies.

If all bidders receive bribes in this fashion, then if $i$ chooses to remain in play and bid optimally, his expected surplus depends on how many, and in asymmetric cases which, other bidders accept these bribes and which remain in play.
%
This in turn depends on bidder $i$'s beliefs about how much each other player was offered.

For simplicity, we will assume that the bribe amounts $z_j$ are all \emph{common knowledge}.
%
We justify this assumption on the following basis:
%
\begin{enumerate}
  \item 
    The only reason the adversary should have to bribe any one bidder to abstain is to prevent the item from being allocated. 
    %
    Therefore, an adversary who attempts to bribe one bidder must attempt to bribe all bidders.
  
  \item
    An adversary is likely to take the same approach to pricing his bribe, i.e.~generously or parsimoniously, for all bidders.
    %
    So if $z_j$ is high, then $z_j$ is likely high for all $j\in N$.

\end{enumerate}

Thus, player $i$'s payoff if he remains in play depends on the expectations
%
\[
  \bar\lambda_j(z_j) := \Expectation[\lambda_j^*(\tilde{v}_j,z_j)]
\]
%
of the optimal mixed strategy $\lambda^*(\tilde{v}_j,z_j)$ played by each opponent $j\in [N]\setminus\{i\}$, based on their unknown type $\tilde{v}_j$ and the known bribe amount $z_j$.
%
Thus a necessary condition for optimality is that $i$ accepts the bribe only when
\begin{align*}
  z_i &\geq \Expectation[U_i(b(v_i,\tilde{v}_{-i})) \mid i\in \alpha(\lambda_{-i}), v_i] \\
  &= \sum_{S\subseteq [N]\setminus\{i\}} c_{i,S}(v_i) \prod_{j\in S}\bar\lambda_j(z_j) \prod_{i\neq k\not\in S} (1-\bar\lambda_k(z_k)).
\end{align*}
%
A non-pure strategy is possible only when this inequality is an equality.


\paragraph{Long-running coalition}
%
A similar formula comes up in the following more believable scenario: a proportion $\lambda\in[0,1]$ of the full scheduler population enters into a long-running coalition to censor items from a class $T$.
%
Assume that items are scheduled by a sequence of repeated auctions, one per epoch, in which $N$ members of the population are randomly sampled to participate.
%
Then the per epoch opportunity cost to a $v$-typed member of the coalition, and therefore the amount that must be compensated by the leader of the coalition, is
\[
  \sum_{k=0}^{N-1}\lambda^{N-1-k}(1-\lambda)^k c_k(v),
\]
%
and censorship succeeds in a proportion $\lambda^N$ of epochs.




We will distinguish two basic classes of examples:
\begin{enumerate}
  \item Common value, i.e.~the $v_i$ are deterministic and known to all participants;
  \item Strictly monotone, i.e.~the function $c_i(v)$ is strictly increasing in $v$ and the distribution of $\tilde{v}$ is without atoms.
\end{enumerate}

\paragraph{Common value case}
Here $\bar\lambda(z)=\lambda^*(z)$ and we can solve for $z$.


\begin{example}[Second price common value auction]

  As we show in an earlier example, the surplus coefficients are $c_0 = v$ and $c_k=0$ for $k>0$.
  %
  If $z>v$, all bidders accept the bribe.
  %
  If $z\in[0,v]$, the bidders pay a mixed strategy defined by the equation $z=v\lambda^{N-1}$; that is,
  \[
    \lambda^*(z; v) = (z/v)^{1/(N-1)}.
  \] 
  We note that it is a strictly increasing and concave function of $z$, and strictly concave if $N>2$.
  %
  If $N=2$, we have $\lambda^*(z)=z/v$.
  
  Finally, we note here that the logarithmic derivative is 
  \[
    \frac{d\log \lambda^*}{dz}(z)= ((N-1)vz)^{-1}.
  \]

\end{example}


\paragraph{Noisy value case}
The noise assumption means that 
\[
  \Prob[\lambda_i(\tilde{v_i},z_i)\in(0,1)]=\Prob[z_i=c_i(\tilde{v}_i)] = 0
\]
for all $z_i$. 
%
We therefore obtain the identity
\begin{align*}
  \bar\lambda_i(z_i) &= \Prob[\lambda_i(\tilde{v_i},z_i) = 1] \\
  &= \Prob[c_i(\tilde{v}_i,z_{-i}) < z_i ] \\
  &= F_{c_i(\tilde{v}_i,z_{-i})}
\end{align*}
where $F_{c_i(\tilde{v}_i,z_{-i})}$ is the CDF of $c_i(\tilde{v}_i,z_{-i})$.
%
By hypothesis, this function is continuous and monotone increasing in $z_i$.
%
By applying this fact to the expansion of $c_i(v_i,z_{-i})$, we find that the latter is also continuous and monotone increasing in $z_{-i}$.

\begin{itemize}
  \item If bidder values are close, we can discount $c_k(\tilde v)$ for $k>0$.
  \item If the auction has a reserve price $R$, then a typical structure is that if just one bidder participates, he pays $R$ and receives the item (or receives the max fee $R$ and provides the service, in the case of a procurement auction), earning a surplus of $\tilde v - R$.
  \item Under both of these assumptions, the above can be reduced to a question of the distribution of $\tilde v$.
  \item Suppose $\tilde v$ is distributed according to a power law on $[0,1]$, and $R=0$. So $F_{\tilde v}(z)= Cz^{\alpha}$ for some $C>0$ and $\alpha>0$ (in fact, $C=\alpha+1$).
\end{itemize}

In the symmetric case, we have $\Expectation\lambda(z)=0$ (i.e.~no one accepts the bribe) if and only if
\[
  \Prob[c_{N-1}(\tilde v)<z) = 0.
\]
This condition is probably not very realistic in practice.
%
So what we should really be looking for is to keep $\Expectation\lambda(z)^N$ as small as possible for $z$ within a reasonable range.

\begin{question}What happens when obtain large $N$ by adding a long tail of bidders with low $\tilde{v}$?\end{question}

\begin{example}[Monopoly profits]

  Suppose that bidder values are very close, so that $\Expectation c_{i,k}(\tilde v_i)$ is small for all $k>0$.
  %
  Then we can approximate
  \[
    \Expectation c_i(\tilde v_i) \approx \Expectation c_{i,0}(\tilde v_i) \cdot \Expectation \lambda(\tilde v)^{N-1}.
  \]
  For simplicity, let's work for now in a model where $\Expectation c_{i,k}(\tilde v_i)$ is exactly zero.\footnote{Note that by revenue equivalence, such a mechanism cannot be efficient. It should not allocate the item if $k>0$, which isn't very realistic.}
  %
  Then $\Expectation\lambda(\tilde v,z)$ solves the equation
  \[
    x = \Prob[\tilde{c}_0x^{N-1} < z] = F(zx^{1-N})
  \]
  where $\tilde{c}_0$ is the monopoly profit and $F$ is its CDF.
  %
  Then to achieve a censorship probability of $p=x^N$, we must bribe at least
  \[
    z = F^{-1}(x)x^{N-1}.
  \]
  
  Suppose that monopoly profits are distributed according to a power law $F(u) = u^{1/K}$ for $u\in[0,1]$.
  %
  Then the bribe per bidder is
  \[
    z = x^Kx^{N-1} = px^{K-1}.
  \]
  If $K>1$ then a briber can achieve probability $p$ of successful censoring with a total bribe of less than $p\pi$, where $\pi$ is the maximum bribe.
  
  So for example, if $N=10$ and $K=3$ then a briber can achieve a non-negligible probability $p=0.1$ of success with a bribe of $0.063$ times the maximum.
  %
  While not exactly catastrophic, this shows that the requirement $v_\Adversary\geq N\pi$ is not a hard limit.

\end{example}

\begin{example}[Bidder reputation]

  A general rule of thumb is that the more information the adversary has, the cheaper it becomes to bribe.
  %
  For example, consider the case that bidders are partitioned into two sets: a `high reputation' set $[N]$ of bidders that handle most uncensored requests, and a long tail 'low reputation' set $[M]$ that comes into play when the high reputation set has been bought off.
  %
  The high reputation set may be quite small; for example, $N=2$ in PBS today (over 90\% of blocks are handled by just two entities).

  Suppose that types and strategies are symmetric within each of the two sets, and that $v_N>v_M$ almost surely. 
  %
  We are allowed two bribe amounts $z_N$ and $z_M$ and strategic plans $\lambda_N^*(\tilde{v},z_N)$ and $\lambda_M^*(\tilde{v},z_M)$.
  %
  Censorship succeeds with probability $p=\bar\lambda_N(z_N)^N\bar\lambda_M(z_M)^M$.


\end{example}


%%%%%%%%%%%%%%%%%%%%%%%%%%%%%%%%%%%%%%%%%%%%%%%%%%%%%%%%%%%%%%%%%%%%%%%%%%%%%%

\newpage
\subsection{Adversary decision problem}

Suppose the bribing adversary stands to gain a payoff of $v_\Adversary>0$ for successfully preventing allocation of the item.sequ
%
Then the adversary's expected utility function has the form
%
\[
  U_\Adversary(z;\bar{\lambda}) = v_\Adversary\prod_i\bar\lambda_i - \sum_i\bar\lambda_iz_i.
\]
%
In the symmetric case, this becomes 
\[ 
  \Expectation[U_\Adversary(z)] = v_\Adversary\bar{\lambda}(z)^N - Nz\bar{\lambda}(z)
\]
(compare \cite[\S A.6]{FPR}.)

The game then becomes one of bounding below
\[
  z/\bar\lambda(z)^{N-1}
\]
independently of $z$.


\paragraph{Asymptotics of $U_\Adversary$}
In the common value case, $\lambda^*(z)$ solves a polynomial equation $Z(G(z),1-G(z))=z$ of degree $N-1$.
%
Thus we can write
\[
  \lambda^*(z)^{N-1} = \chi(\mathcal{M})^{-1}\left( z - \tilde Z(G(z),1-G(z)) \right)
\]
where $\tilde Z$ is a polynomial of degree at most $N-2$.
%
Note that $\tilde Z$ depends only on $c_k$ for $k>0$; the monopoly surplus $c_0$ only shows up in the formula for the leading coefficient in $Z(p,1-p)$.
%
That is, 
\[
  U_\Adversary(z) = \lambda^*(z)\left[ \left(\frac{v_\Adversary}{\chi(\mathcal{M})} - N\right) z + o(z) \right].
\] 
From this we can see that $U_\Adversary(z)>0$ for sufficiently large $z$ iff $v_\Adversary > N\chi(\mathcal{M})$.

This is only relevant if $c_0$ is also sufficiently large for the asymptotic to kick in.
%
We can typically arrange this by growing the max fee (assuming this doesn't affect the other surplus coefficients).

\begin{lemma}

  In the common value case, $z/\lambda^*(z)^{N-1} \rightarrow \chi(\mathcal{M})$ as $z\rightarrow\infty$.

\end{lemma}

This reduces the study of the inequality $v_\Adversary < N z/\lambda^*(z)^{N-1}$ to a compact set independent of $c_0$.

\paragraph{Differential analysis of $U_\Adversary$}
We have
\begin{align*}
  \frac{dU_\Adversary}{dz}(z) &= N\left[ v_\Adversary\bar{p}'(z)\bar{p}(z)^{N-1} - (\bar{p}'(z)z + \bar{p}(z) ) \right] \\
  &= N\left[ \bar{p}'(z)(v_\Adversary\bar{p}(z)^{N-1} - z) - \bar{p}(z) \right]
\end{align*}
on $(c_{N-1},c_0)$.
%
At the limits we have
\begin{align*}
    \frac{dU_\Adversary}{dz}((c_0)_-) &= N\left[ \bar{p}'((c_0)_-)(v_\Adversary-c_0) - 1 \right] \\
    \frac{dU_\Adversary}{dz}((c_{N-1})_+) &= -c_{N-1}N\bar{p}'((c_{N-1})_+) \leq 0
\end{align*}

\begin{example}[Second price common value]

  We return to the case of the second price auction with common value $v$.
  %
  Recall that $\lambda^*(z)=(z/v)^{1/(N-1)}$ on $[0,v]$.
  %
  Then 
  \begin{align*}
    U_\Adversary(z) &= v_\Adversary (z/v)^{\frac{N}{N-1}} - Nz(z/v)^{\frac{1}{N-1}} \\
    &= z^{\frac{N}{N-1}}v^{-\frac{1}{N-1}}\cdot \left(v_\Adversary/v - N  \right)
  \end{align*}
  for $z\in[0,v]$.
  %
  This utility function is concave (resp.~convex) if and only if it is negative (resp.~positive) for $z>0$.
  %
  The auction is censorship resistant if and only if $v_\Adversary < Nv$ (concave).

  In the case of a procurement auction with max fee $\pi$ and common marginal cost $\mu$ \cite[\S3.2]{roughgarden2024transaction}, the bound becomes $N(\pi-\mu)$.

\end{example}


\newpage
\section{Dynamic scheduler population}

\begin{itemize}
  \item 
    If we find anything here, it should be that the CR calculations are much simpler and more convincing than in the parallel case.
  \item
    The game sequence where schedulers first declare themselves active and then enter into contracts to become inactive ($\in\Population_\delta$) again may seem convoluted, but we will argue that this actually most resembles a realistic situation in the presence of markets of secondary holders of transferable blockspace future.
  \item
    In fact, it already somewhat resembles the situation with proposers outside the lookahead, except that these proposers do not directly control their entry into the active set $\Population_\alpha$.
  \item
    For simplicity we assume equal weighted bribes are used, but in reality this approach is probably suboptimal for the asymmetric sequential arrivals model.
    %
    Some cooperative game theory should be used to do it properly.
  \item
    The focus on posted price mechanisms (including with a `fire sale' at the end) is not artificial: it is known that in the cases of interest, these implement efficient allocations.
\end{itemize}

We now anchor ourselves in a dynamic setting in which would-be schedulers must first declare themselves as \emph{active}, incurring a cost $\kappa>0$, before they may participate in the blockspace supply mechanism --- whether by bidding to schedule items or by entering into contracts to abstain.
%
The scheduler population $\Population$ may be infinite, but only finitely many schedulers may declare themselves active at any one time.
%
Denote by $\Population_\alpha$ the active population.

Each would-be scheduler decides whether to declare themselves active, given information available at time $t$, if their expected surplus from participation exceeds the registration cost, i.e.~if
\[
  c_t(v) > \kappa.
\]
We suppose it becomes common knowledge when an item has been scheduled.

\begin{example}[Sequential arrival of short-lived schedulers]

  Suppose that exactly one scheduler may declare itself active at each epoch.
  %
  At the end of the epoch, the active scheduler becomes permanently inactive, i.e.~enters $\Population_\delta$.
  %
  Thus schedulers do not need to consider the value of withholding blockspace for allocation at a later epoch.
  
  Items are advertised with a posted fee $v^*(t)$ (which may depend on the epoch).
  %
  In each epoch, if a scheduler $a$ is active and $v_a<v^*(t)$, the item is allocated to $a$, who receives the fee $v^*$ and retains surplus $v^*-v_a$.
  %
  The (deterministic) surplus for the scheduler that may enter in epoch $t$ is
  \[
    c_t(v) = v^*(t)-v_a
  \]
  if $v^*(t)<v_a$ and $0$ otherwise.
  %
  Entry is feasible iff $v^*(t)-v_a > \kappa$.

  The censor's only strategy is to contract directly with the scheduler in epoch $t$ when it appears at the start of each epoch, at which point the price to beat is $v^*(t)-\tilde{v}_a$.
  %
  To achieve a probability $p>0$ of successful censorship over $N$ epochs, the adversary must pay
  \[
    z(t) = v^*(t) - F_{\tilde{v}}^{-1}(p^{1/N})
  \]
  to the scheduler in epoch $t$, where $F_{\tilde{v}}^{-1}$ is the quantile function of $\tilde{v}$.
  %
  If the posted price $v^*$ is constant, the total cost simplifies to
  \[
    CR = N(v^*-F_{\tilde{v}}^{-1}(p^{1/N})).
  \]
  
\end{example}

\begin{example}[Sequential arrival of short-lived schedulers, static population]

  Suppose allocation works as in the above example, except that the population of schedulers and their assignment to epochs is common knowledge at the start of play.
  %
  Then the $n$th scheduler has an expected surplus of
  %
  \begin{align*}
    c_n(v) &= \Prob[\tilde{v}_k < v^*, k=0,\ldots,n-1] \cdot (v^* - v) \\
    &= F_{\tilde{v}}(v^*)^{n-1}\cdot (v^* - v).
  \end{align*}
  %
  The total cost to subsidise a grand coalition with probability $p$ of success (with symmetric payouts) is
  \begin{align*}
    CR &= \sum_{n=0}^{N-1} F_{\tilde{v}}(v^*)^{n}\cdot (v^* - F_{\tilde{v}}^{-1}(p^{1/N})) \\
    &= \frac{1- F_{\tilde{v}}(v^*)^N}{1-F_{\tilde{v}}(v^*)} (v^* - F_{\tilde{v}}^{-1}(p^{1/N})).
  \end{align*}
  %
  This number is bounded above by $1/(1-F_{\tilde{v}}(v^*))$ as $N\rightarrow\infty$.
  %
  For example, if $v^*$ is chosen at the median of the type distribution so that $F(v^*)=1/2$, the maximum censorship resistance factor is just $2$.

  Now suppose the arriving schedulers know their assignment to epochs in advance, and may \emph{choose} whether or not to make themselves publicly known at the start of the period.
  %
  If $v_\Adversary > N(v^* - v(p^{1/N}))$, then the scheduler can arrange a larger bribe for himself by choosing not to reveal himself until the epoch in which he becomes active.
  %
  On the other hand, if the adversary's budget would cover a grand coalition on a static population declared at the start but not the full cost of a dynamic approach, the scheduler at the back of the queue may be better off declaring himself in advance.

\end{example}

\begin{example}[Sequential arrival of long-lived schedulers]

  If bidders are long-lived, they may have multiple opportunities to bid for an item.
  %
  Conversely, prospective new entrants must consider whether they will face competition from incumbent schedulers who haven't yet received an allocation.

  If the item is not scheduled in a given epoch $t$, it may be inferred that no scheduler with cost $v_i<v^*$ was active during that epoch.
  %
  It follows that $\Population_{\alpha,t} = \Population_{\delta,t}$, and as-yet-inactive members of $\Population$ may estimate their entry surplus $c_{t+1}(v)$ as though they are entering a pristine competition environment.

  Suppose that registration rates are throttled so that at most $N$ new schedulers may register at any time.
  %
  (How are the $N$ new schedulers selected? We haven't allowed them to differentiate based on registration cost $\kappa$.)
  %
  The short answer is that the more schedulers get selected

\end{example}

\printbibliography