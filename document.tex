\maketitle
%\thispagestyle{fancy}

\begin{abstract}

  Censorship in blockchain transaction processing systems occurs when an adversary intentionally arranges for exclusion of transaction items.
  %
  Since the main objective of blockchain systems is to arrange for transaction \emph{inclusion}, successful censorship is an economic inefficiency that can signal the exercise of market power.
  %
  Hence, it is natural to try to combat censorship by introducing competition on the supply side of the blockspace market.

  We posit, and interrogate, a simplistic \emph{product rule} for censorship resistance: if a transaction could be confirmed by any of $N$ active schedulers, then a transaction fee limit increment of $\delta$ causes the cost of censorship to increase by $N\delta$.
  %
  We find that the product rule is fairly robust to changes in the scheduling mechanism, but fails completely if the censoring adversary is given the power to form atomic coalitions; this includes some situations that seem potentially realistic.
  %
  The rule recovers if we introduce dynamic scheduler populations, yielding a censorship-resistance argument in favour of secondary markets for blockspace futures.
  
\end{abstract}

\section{Introduction}

In blockchain systems, censorship is generally characterised as a deliberate attempt to prevent or delay a transaction or class of transactions from being posted to the chain.
%
Broadly speaking, a blockchain is \emph{censorship resistant} if censorship is `hard' to achieve.
%
Censorship resistance is often considered an essential feature of permissionless public blockchains, especially for the functioning of financial primitives \cite{buterin2015problem}.

In principle, the censorship resistance of a blockchain can be quantified as the minimum cost to achieve censorship by any means.
%
However, because of the diversity of available means --- which may target various layers of the technology stack and have different domains of effectiveness --- the estimation of these costs poses a substantial challenge \cite{wahrstatter2024blockchain}.
%
To attempt this calculation, we must distinguish and study the costs associated with each censorship strategy in context.

Tautologically, all means to effect censorship of a class $T$ of transaction items fall into one of the following two categories:
%
\begin{enumerate}
  \item Arrange that no schedulers are able to handle items in $T$ (e.g.~by hiding them via an eclipse attack \cite{heilman2015eclipse}).
  \item Arrange that even if schedulers could handle items in $T$, they choose not to.
\end{enumerate}
%
The second class of approaches is precisely the class that can be analysed in terms of economic incentives, and is the focus of this paper.

To achieve censorship through incentive manipulation, a censoring adversary must present all schedulers that could handle $T$ with information that either causes them to revise their expected payoffs for handling $T$ or unlocks new actions that are more attractive and mutually exclusive with $T$.

The most direct way to achieve this --- and the most likely to be analytically tractable --- is by presenting the scheduler with a contract that triggers a value transfer in the event that he does not schedule elements of $T$.
%
The value transfer must compensate the scheduler for the profit he forgoes by omitting $T$; the cheapest possible such contract is one that \emph{only} requires him to censor $T$, and does not exclude any other potentially lucrative opportunity.

We focus on this case because it is tractable and because it has attracted recent attention \cite{FPR} that has led to proposals to modifying Ethereum's scheduling environment under the guise of censorship resistance \cite{burian2024censorship}.

\subsection{Contracts for exclusion}
\paragraph{Censorship at the blockspace supply mechanism layer}
The focus of this paper is censorship resistance of the \emph{blockspace supply mechanism}, that is, the mechanism governing the allocation by \emph{schedulers} of blockspace to transaction items in exchange for payments.

More precisely, we estimate the costs of a particular censorship strategy via \emph{contracts for exclusion}, that is, payment arrangements with schedulers conditioned on the non-inclusion of some specified class $T$ of transaction items \cite{FPR}.
%
In such a strategy, an adversary attempts to secure agreements from all schedulers that would be in a position to handle items in $T$ during a target period that they will instead sit out of the market for $T$ throughout that period.

Payment arrangements can take the form of bribes, where the scheduler is paid if he obeys the rule, and threats --- such as the notorious OFAC sanctions \cite{wright2022defi} --- where a cost imposed on the scheduler if he does not.
%
These two approaches differ by the payment amounts and directions, but not on the condition; we consider them both on equal footing.



\paragraph{Defining censorship cost}
%
A rational scheduler will not enter into a contract for exclusion unless the terms of the contract provide that he be compensated for his opportunity cost in sitting out of the market.
%
Exactly how much compensation is required, and hence the cost to an adversary of offering such terms, depends essentially on the fine details of the environment and the decision problem it presents to the scheduler:
%
\begin{enumerate}

  \item 
    The mechanics of the scheduling pipeline, that is, how decisions about blockspace allocation are finalised;
  \item 
    Details of the targeted items such as the maximum fee;
  \item
    The variable costs, preferences, and beliefs of the scheduler, his view of other opportunities that complement or conflict with the targeted items, and his capacity to exploit them.
\end{enumerate}
%
Moreover, the third item --- the scheduler's \emph{type} --- is generally not known to the adversary, who therefore cannot compute ex-ante the minimum compensation the scheduler would accept.
%
Instead, the adversary's beliefs about his counterparty's type forms some probability distribution, and the best he can do is make an offer that will succeed with probability at least as great as some threshold $0<p<1$ found to be tolerable.

While the system planner may have agency over some of the details of the scheduling algorithm, in general he cannot directly control the other two items.
%
Thus, it is not possible to describe the cost of censorship associated to a particular design choice, or even a design choice plus a transaction item, by a single number: it is a function over the product of the unit interval, parametrising the threshold, and a potentially complicated space of scheduler types.




\paragraph{Multiple schedulers}
%
In public blockchains, the responsibility of deciding which transactions to add to the schedule rotates among multiple independent schedulers from epoch to epoch.
%
Assuming transactions cannot be sequenced multiple times, these schedulers are effectively in competition with each other for order flow.
%
An adversary that wishes to censor transactions for multiple epochs must therefore make arrangements with multiple competing schedulers.
%
Estimating the cost of doing so must therefore take into account the types of these various schedulers and their strategic interactions.

In this paper, we argue that the full details of scheduling environments with multiple schedulers whose types may differ are captured by the formalism of mechanism design \cite{milgrom2004putting}.
%
In this formalism, the scheduling algorithm, value of the targeted items, and private information of the scheduler, are captured by the outcome function, payoff functions, and type distributions of a strategic form mechanism with schedulers as players.
%
If we endow the strategy spaces $\Action$ of our mechanism with a distinguished point $\bot$ representing `do nothing,' then a contract for exclusion is nothing more than an offer of payment to each player to select $\bot$ as their strategy.

\begin{example}

  The rotating block producer mechanism can be modelled as sequentially arriving short-lived bidders; see \cite[Chap.~2]{gershkov2014dynamic}.

\end{example}

\begin{remark}[Negotiation]

  For simplicity, in this paper we mostly assume that the type distributions of schedulers are common knowledge, and that moreover before contracting there is no negotiation phase through which the adversary could elicit additional private information about the type of his counterparty: he must simply make a take-it-or-leave-it offer based on the common prior.

\end{remark}


There are also systems that allow multiple schedulers to participate in constructing the block over the course of a single epoch:
%
\begin{itemize}

  \item 
    In Filecoin's expected consensus system, a random number of consensus nodes are elected each epoch to construct a block; the resulting blocks are concatenated and deduplicated.

  \item
    In Ethereum's proposer-builder separation,

  \item
    Recent proposals for imposing constraints on blocks to force the inclusion of certain transactions allows the creators of these lists to act as schedulers in addition to the builder of the main block.

  \item
    Ideas for multiple concurrent proposers have been floated to increase censorship resistance \cite{FPR}.

\end{itemize}
%
The formalism of contracts for exclusion allows us to study censorship resistance quantities under these, or any other mechanism, on an equal footing.


\paragraph{Scaling censorship resistance}

The rotating scheduler architecture of public blockchains such as Bitcoin and Ethereum is generally thought to increase the difficulty of censoring transactions over a period of many blocks.
%
Indeed, if we allow ourselves to assume that every block over the period of interest is made by a different scheduler, it is quite easy to construct an informal argument as to why the cost of censorship scales linearly in the number of blocks: in this case, each scheduler's payoff depends only on its actions and the actions of those that came before it, which it cannot influence;
%
(See \cite[\S6.1]{FPR} for a special case of this.)


By extension of that same reasoning, recent proposals \cite{FPR,2024forkchoice} have sought to improve the single-slot censorship resistance (or `neutrality' \cite{ma2024uncrowdable}) properties of the Ethereum chain by increasing the number of parties capable of allocating blockspace in each slot.

\begin{example}[Multiple schedulers, pay as bid]

  Let $T$ be a single transaction with fee limit $\tau$ and posit are $N$ schedulers, each with vanishing marginal costs.
  %
  Each scheduler could in principle earn a profit of $\tau$ by handling $T$; therefore a successful coalition must offer each scheduler at least $\tau$ for abstaining.
  %
  This puts the total cost of censorship (at $p=1$) at $N\tau$.
  %
  This is the intuition behind the \emph{product law}, which poses that the cost of censorship is linear in the number of schedulers and the transaction fee limit.

\end{example}




\subsection{Censorship and efficiency}

Contrast spam and application layer censorship \cite[Def.~4]{wahrstatter2024blockchain}, which do collateral damage in the sense of also being mutually exclusive with transactions outside the targeted flow.

\paragraph{Censorship and efficiency}
Censorship by contracts for exclusion is distinguished from other approaches to censorship by the fact that it results in an economically inefficient allocation of blockspace --- namely, one in which free blockspace fails to be allocated even though there are available transaction items that offer a fee in excess of the processing costs of some scheduler that could handle it.
%


By way of contrast, an outcome in which blocks are filled with spam that pays higher transaction fees than targeted `legitimate' transaction material is efficient.
%
That is, from the perspective we adopt in this paper, such outcomes are a case of the blockspace supply mechanism functioning as intended.
%
Meanwhile, attacks that focus on manipulating some other layer (such as the network) proceed by modifying the \emph{environment}, for example by limiting the set of transactions visible to each scheduler.
%
They have no bearing on the efficiency of the method by which transactions are selected from that environment for inclusion.


\begin{remark}[Efficiency angle]

  The idea that introducing multiple independent schedulers can improve the economic efficiency of transaction processing in a blockchain network aligns with general microeconomic intuition: if one scheduler is unable or unwilling to handle a given item, another stands ready to offer the service in his stead.
  %
  Taking the idea to its limit, in an idealised market of perfect competition among schedulers, the inefficiency effected by censorship cannot exist.

  Note that even under perfect competition, other censorship strategies such as spam remain equally feasible.

\end{remark}


\subsection{Previous work}
Prior art on defining censorship and censorship resistance in blockchains and other information systems:
\begin{itemize}
  \item 
    \cite{wahrstatter2024blockchain} captures the \emph{intentional} angle and focuses on inclusion, rather than ordering.
    %
    The authors identify several of the approaches to censorship discussed here \cite[\S4.3]{wahrstatter2024blockchain}, notably application layer censorship and various consensus layer faults (e.g.~failure to attest).
    %
    Its main example appears to be the OFAC sanctions.

  \item
    Some earlier authors who cite the importance of the decentralisation of the scheduler set to censorship resistance: \cite{gencer2018decentralization,silva2020impact}.

  \item
    Vitalik \cite{buterin2015problem} mentions the LIBOR manipulation as typical of the type of `manipulative' coalition that censorship resistance might aim to prevent.

  \item
    Earlier work on censorship in document retrieval systems \cite{perng2005censorship} focuses on the ability of an adversary to inhibit retrieval of information, i.e.~reads, rather than adding new information to a database.
    %
    This work highlights the importance of distinguishing selective censorship from total blackout.
  
  \item
    Some authors \cite{danezis2004economics,FPR} attempt to establish motivations for a censoring adversary; we prefer to black box it.

  \item The rotating block producer story is widely understood to be important in ensuring the timeliness of transaction inclusion. The idea is implicit, though not really interrogated, in the Bitcoin whitepaper \cite[\S5]{nakamoto2008bitcoin}: \emph{New transaction broadcasts do not necessarily need to reach all nodes. As long as they reach many nodes, they will get into a block before long.}

  \item Concerns about delays in transaction inclusion surface in the Bitcoin literature under the guise of `efficiency.' \cite{pappalardo2018blockchain}. Concerns about deliberate censorship do not seem to be widely discussed in the earlier literature.
\end{itemize}



%%%%%%%%%%%%%%%%%%%%%%%%%%%%%%%%%%%%%%%%%%%%%%%%%%%%%%%%%%%%%%%%%%%%%%%%%%%%%%%
\newpage

\section{Model}


Consider a mechanism in which a population $\Population$ of schedulers competes to sequence a transaction item.
%
Exactly one $x$ of the active schedulers will be selected to handle the item, incurring a cost $\theta_x\geq 0$ which is known $x$ at decision time.
%
Each scheduler $i$ receives a fee $f_i(s)$ --- under classical auction formats, $f_i=0$ unless $i$ handles the item.

We will study situations in which an adversary $\Adversary$ seeks to enter into an agreement with each scheduler such that scheduler $i\in\Population$ receives a `bribe' $z_i\geq 0$ if and only if he chooses to abstain from the auction for the targeted item.







\begin{definition}[Cost of censorship]

  The \emph{cost to censor} $\CC{p}(\Mechanism)$ a blockspace supply mechanism $\Mechanism$ at chance bound $p\in[0,1]$ is the smallest total bribe that can be offered to schedulers in order to achieve an overall probability of at least $p$ that the target items are excluded.

\end{definition}

In this definition, details of the environment that affect the cost of censorship are absorbed into the definition of the mechanism $\Mechanism$.
%
For example, transaction fee limits form part of the definition of the allocation rule of $\Mechanism$, while processing costs or opportunity costs of schedulers are incorporated into the distribution of preference types of the scheduler population.

Suppose the bribing adversary stands to gain a payoff of $v_\Adversary>0$ for successfully preventing allocation of the item.
%
Suppose $\Adversary$ pursues a strategy that induces schedulers to exclude the targeted flow with a total probability of $p\in[0,1]$.
%
Then the best possible expected payoff that $\Adversary$ can achieve is
%
\[
  U_\Adversary(z;\bar{\lambda}) = pv_\Adversary - \CC{p}(\Mechanism).
\]
%
A rational, risk-neutral adversary will therefore do nothing as long as $\CC{p}(\Mechanism)/p > v_\Adversary$.

This leads us to make the following definition:
%
\begin{definition}[Censorship resistance]

  A mechanism $\Mechanism$ is $K$-\emph{censorship resistant} if $\CC{p}(\Mechanism)/p>K$ for all $p\in(0,1]$.

\end{definition}

\begin{example}[Single scheduler, pay as bid]

  Suppose $T$ is a single transaction having an attached transaction fee limit $\tau$, and let there one scheduler having marginal cost $\tilde{\theta}>0$ for handling $T$.
  %
  If the scheduler handles $T$, he receives the full fee payment $\tau$, for a profit of $\tau-\tilde{\theta}$.
  %
  The cost to censor is $\CC{p}=\tau+Q_{-\tilde{\theta}}(p)$, where $Q_{\tilde{\theta}}:[0,1]\rightarrow\R$ is the quantile function of $\tilde{\theta}$.

\end{example}



\begin{theorem}[Product law for sure censorship]

  Let $\Mechanism_\tau$ be a family of efficient single item auctions with reservation price $\tau$.
  %
  Suppose schedulers costs $\tilde\theta_i$ are i.i.d, bounded below, and independent of $\tau$.
  %
  Then $\CC{1}(\Mechanism_\tau)<\infty$, and for sufficiently large $\tau_0\gg 0$,
  \[
    \frac{\partial}{\partial\tau}\CC{1}(\Mechanism_\tau) \equiv N
  \]
  %
  where $N$ is the number of schedulers.

  If there exist schedulers with arbitrarily small costs, then $\CC{1}(\Mechanism_\tau) = N\tau$.
  
\end{theorem}
%
\begin{proof}

  A single item auction with reserve $\tau$ is efficient if the item is allocated to the bidder with highest type if this type is greater than $\tau$; otherwise the item is withheld.
  %
  We don't need to place any restrictions on the family other than the bidder type distributions remain constant, though it is natural to consider a family whose description has a natural parameter $\tau$ such as a traditional first or second price auction.
  %
  The argument proceeds by Myerson's payoff equivalence theorem \cite[Thm.~3.3]{milgrom2004putting}, which states that the expected payoffs for each bidder depend only on the allocation rule as a function of costs --- here fixed by efficiency --- and the expected payoff at some reference point in type space, which here we may take as the greatest lower bound of the support of the distribution function of scheduler costs.

  For any mechanism $\Mechanism$, we have $\CC{1}(\Mechanism)=Nz^*_\Mechanism(1)$, where
  \[
    z^*_\Mechanism = \inf\left\{z\in\uR \mid z \underset{a.s.}{\geq} \Util_0(\tilde{\theta}) \right\};
  \]
  that is, $z^*_\Mechanism$ is the smallest bribe that exceeds the payout $\Util_0(\tilde{\theta})$ expected by a monopsony bidder with probability $1$.
  %
  Thus we are reduced to showing that $\partial z^*/\partial\tau (\Mechanism_\tau) \equiv 1$.

  If $\tau \underset{a.s.}{\leq} \tilde\theta$ then the item is never allocated and $\CC{1}(\Mechanism_\tau)=0$.
  %
  Assume $\Prob[\tau > \tilde\theta]>0$, which is true for sufficiently large $\tau$.

  Now, if $\theta\leq \tau$ and $\theta \underset{a.s.}{\leq} \tilde\theta$, then by efficiency a bidder with type $\theta$ always wins the item.
  %
  That is, $\Prob[i\text{ wins}  \mid \tilde{\theta_i} = \theta] = 1$ and so $\Util_0(\theta) = \Expectation_B[\tilde{f}] - \theta$, where $\tilde{f}$ is the fee received by the winning bidder and $\Expectation_B$ is the expectation operator conditioned on the information available to that bidder.
  %
  By the payoff equivalence theorem, $\Expectation_B[\tilde{f}\mid\tilde{\theta}=\theta] = \tau$, since this is what the winning bidder pays in a second price auction with reserve fee $\tau$ and the latter is also efficient.
  

  Let $\theta_0=\sup\{\theta\in\R \mid \Prob[\tilde{\theta}\leq\theta ]= 0\}<\tau$.
  %
  In other words, $\theta_0$ is the greatest lower bound of the support of the distribution function of $\tilde{f}$.
  %
  It follows from Myerson's Lemma that $\Util_0(\theta)$ is a continuous function of $\theta$.
  %
  Thus $z^*=\Util_0(\theta_0) = \tau-\theta_0$.
  \qedhere
 

\end{proof}


\subsection{Bounding censorship cost}

To bound the cost of censorship $\CC{p}$ above, it is enough to exhibit an offer $\vec{z}$ that achieves censorship with probability at least $p$.
%
Then $\CC{p}\leq \sum_iz_i$.


\begin{proposition}[Upper bounding censorship cost with the distribution function]

  Suppose $\Mechanism$ is invertible.
  %
  The following conditions on an offer $\vec{z}\in\uR^N$ are equivalent:
  %
  \begin{enumerate}
    \item 
      $F_{\Util_i(\tilde\theta_i,z_{-i})}(z_i) \geq \lambda_i$ for each $i=1,\ldots, N$.

    \item
      $\CC{\prod_{i=1}^N\lambda_i}(\Mechanism) \leq \sum_{i=1}^N z_i$.
  \end{enumerate}

\end{proposition}

This means in particular that if scheduler margins are tight with high probability, that is, costs for most schedulers are clustered near to the max fee $\tau$, then the cost to censor can be reasonable even if schedulers occasionally achieve high margins.

\begin{corollary}

  Suppose $\Mechanism$ is invertible.
  %
  Then $\CC{p}(\Mechanism) \leq N\cdot Q_{\Util(\tilde\theta,z)}(p^{1/N})$.

\end{corollary}

\begin{corollary}[Cost to censor in a market with tight scheduler margins]

  Suppose that scheduler costs are $(q,\epsilon)$-concentrated near $\tau$, i.e.~$\Prob(|\tau-\tilde\theta| \leq \epsilon) \geq q$.
  %
  Then $\CC{q^N}(\Mechanism)\leq N\epsilon$.

\end{corollary}

To bound $\CC{p}$ below, say by some $K$, we must show that for any $\vec{z}$ satisfying the constraint
%
\[
  \prod_i\Prob[\Util (\tilde{\theta}_i,z_{-i}) \geq z_i] \geq p 
\]
%
we must have $\sum_iz_i \geq K$.
%
This seems to be more difficult in general because we must consider the entire region of feasible offers, rather than simply exhibiting a single element.


In order to apply the simplified analysis available for symmetric offers to obtain lower bounds, we would need to know that a symmetric offer always does at least as well as an asymmetric one.
%
However, this is not the case:





\begin{comment}
Some examples:
%
\begin{itemize}
  \item Network layer censorship --- eclipse attack, spam.
  %
  \item Consensus level Byzantine equivocation, failure to attest. Thresholds here are guaranteed by results in BFT consensus theory and apply to the full consensus value, i.e.~full blocks in Ethereum.
  %
  \item System state manipulation (e.g. governance attack, randomness manipulation). This includes what \cite{wahrstatter2024blockchain} calls \emph{application layer} censorship.
  %
  \item Blockspace congestion.
\end{itemize}
%

\begin{example}[Liveness failures without censorship]

  In some cases, it may be incentive compatible to omit transactions without the intervention of an adversary with preferences over exclusion.
  %
  For example, if the fee attached to a transaction is less than the marginal orphan risk increase associated to its inclusion, it should be individually rational to omit it; the costs associated with the previous calculation therefore vanish.
  %
  If these types of situations fall within our definition of censorship, then we are forced to accept that the associated `censorship resistance' values are zero (or negative).\footnote{Such cases do fall into scope for some objectives quoted for Ethereum, such as `chain neutrality' \url{https://ethresear.ch/t/uncrowdable-inclusion-lists-the-tension-between-chain-neutrality-preconfirmations-and-proposer-commitments/19372/6}.}

\end{example}

\begin{example}[Censorship by congestion]
  
  Suppose a censoring adversary can perform only the actions assumed of the `honest' users of our blockchain system: creating transaction items and offering payment in exchange for their inclusion.
  %
  Such an adversary may attempt a \emph{spam attack}, creating junk transactions with sufficiently high fee bids that, during the attack, no fee-maximising block contains the targeted items.
  %
  Do we need to take into consideration attacks of this form in computing censorship resistance?
  
  I will argue that a blockspace supply mechanism subject to a successful attack of this form is essentially functioning as intended.
  %
  The optimal allocation of blockspace to items with sufficient payments is an economically efficient outcome, regardless of the intent behind the creation of those items.
  %
  I'll use this argument to justify not attempting to compute the costs of such an attack, \emph{even if it might be cheaper} than other approaches (though the reader may find this rather unlikely, we can construct examples in which it would be).
  
\end{example}
\end{comment}




\subsection{Scheduler decision problem}

Suppose that after bidder $i$ draws his type $\theta_i$, but before other actions are taken, an adversary $\Adversary$ offers $i$ a bribe $z_i$ to abstain from the auction.
%
Bidder $i$ has two pure strategies at this stage:
\begin{enumerate}
  \item Reject the bribe and play his optimal strategy in the rest of the game.
  \item Accept the bribe and abstain from play, netting $z_i$.
\end{enumerate}
%
More generally, $i$ may choose an arbitrary mixture $\lambda_i(\theta_i,z_i)\in[0,1]$ of the two strategies.

If all bidders receive bribes in this fashion, then if $i$ chooses to remain in play and bid optimally, his expected surplus depends on how many, and in asymmetric cases which, other bidders accept these bribes and which remain in play.
%
This in turn depends on bidder $i$'s beliefs about how much each other player was offered.

For simplicity, we will assume that the bribe amounts $z_j$ are all \emph{common knowledge}.
%
We justify this assumption on the following basis:
%
\begin{enumerate}
  \item 
    The only reason the adversary should have to bribe any one bidder to abstain is to prevent the item from being allocated. 
    %
    Therefore, an adversary who attempts to bribe one bidder must attempt to bribe all bidders.
  
  \item
    An adversary is likely to take the same approach to pricing his bribe, i.e.~generously or parsimoniously, for all bidders.
    %
    For example, if $z_i$ is high enough to make bidder $i$ consider accepting with high probability, then $z_j$ is likely similarly high for all $j\in N$.

\end{enumerate}

Thus, player $i$'s payoff if he remains in play depends on the expectations
%
\[
  \bar\lambda_j(z_j) := \Expectation[\lambda_j^*(\tilde{\theta}_j,z_j)]
\]
%
of the mixed strategy $\lambda^*(\tilde{\theta}_j,z_j)$ played by each opponent $j\in [N]\setminus\{i\}$, based on their unknown cost type $\tilde{\theta}_j$ and the known bribe amount $z_j$.

Thus the condition for optimality is that $i$ accepts the bribe whenever
\begin{align*}
  z_i &> \Expectation[U_i(\theta_i,\tilde{\theta}_{-i})) \mid i\in \Population_\alpha(\lambda_{-i}), v_i] \\
  &=: c_i(\theta_i,z_{-i}) \\
  &= \sum_{S\subseteq \Population\setminus\{i\}} c_{i,S}(\theta_i) \prod_{j\in S}\bar\lambda_j(z_j) \prod_{i\neq k\not\in S} (1-\bar\lambda_k(z_k))
\end{align*}
%
and rejects it whenever the reverse strict inequality is satisfied.
%
A non-pure strategy is possible only when this inequality is an equality.

\begin{proposition}[Characterisation of equilibria]
  \label{thm:bidder-equilibrium}

  A strategy profile $(\lambda_i^*)_{i=1}^N$ is a Nash equilibrium of the abstention game $(\Mechanism,\vec{z})$ if and only if
  %
  \[
    \lambda_i^*(\theta_i,\vec{z}) = \left\{\begin{array}{ll} 
      1 & \text{if } z_i > \Util_i(\theta_i,z_{-i}) = Z_i(\theta_i,\bar\lambda(z_{-i})) \\ 
      0 & \text{if } z_i < \Util_i(\theta_i,z_{-i}) = Z_i(\theta_i,\bar\lambda(z_{-i}))
    \end{array}\right.
  \]

\end{proposition}

\begin{corollary}[Existence of equilibria from expected play]
  \label{thm:expected-equilibria-existence}

  A vector $(\lambda_i)_{i=1}^N$ is the vector of expectation values of a Nash equilibrium if and only if the inequalities
  %
  \[
    \Prob[Z_i(\tilde\theta_i,\lambda_{-i}) < z] \leq \lambda_i \leq \Prob[Z_i(\tilde\theta_i,\lambda_{-i}) \leq z]
  \]
  %
  are satisfied.

\end{corollary}
%
\begin{proof}
  
  Suppose given a set $\lambda_i$ satisfying the desired inequalities.
  %
  Let $\Theta_i(z_i)=\{\theta\in\Theta_i \mid Z_i(\theta,\lambda_{-i}) = z_i\}$.
  %
  It suffices to construct a function $\lambda^*_i:\Theta_i(z_i)\rightarrow[0,1]$ such that
  %
  \[
    \lambda_i - \Prob[Z_i(\tilde\theta_i,\lambda_{-i}) < z] = \int_{\Theta_i(z_i)} \lambda_i^*(\theta)dF_{\tilde\theta}
  \]
  %
  for each $i$.
  %
  Since 
  %
  \[
    0 \leq \lambda_i - \Prob[Z_i(\tilde\theta_i,\lambda_{-i}) < z] \leq \Prob[Z_i(\tilde\theta_i,\lambda_{-i}) \leq z] - \Prob[Z_i(\tilde\theta_i,\lambda_{-i}) < z] ]= \Prob[\tilde\theta_i\in \Theta_i(z_i)]
  \]
  %
  this is clearly possible.
  %
  \qedhere

\end{proof}

\begin{corollary}[Uniqueness of equilibria from expected play]
  \label{thm:expected-equilibrium-uniqueness}

  A Nash equilibrium strategy profile is uniquely determined away from $\Util_{i,\vec{z}}^{-1}(z_i)$ by its vector of expectation values $\bar\lambda_i(\vec{z})=\Expectation[\lambda_i^*(\tilde\theta_i,\vec{z})]$.
  %
  If $\Util_i(-,\vec{z})$ is strictly monotone increasing, then it is uniquely determined everywhere.

\end{corollary}
%
\begin{proof}

  The first claim is immediate from Proposition \ref{thm:bidder-equilibrium}.
  %
  For the second, let $\theta_i\in\Theta_i$ be the unique type such that $\Util_i(\theta_i,z_{-i})=z_i$.
  %
  Then
  %
  \begin{align*}
    \bar\lambda_i  &= \Expectation[\lambda^*_i(\tilde\theta_i)] \\
    &= \lambda^*_i(\theta_i)\cdot\Prob[\tilde\theta_i=\theta_i] + \Prob[\tilde\theta_i>\theta_i]
  \end{align*}
  %
  which can be rearranged to give a formula for the only remaining value $\lambda^*_i(\theta_i)$. 
  %
  \qedhere

\end{proof}

By abuse of language, we will call a vector of probabilities $(\lambda_i)_{i=1}^N$ an \emph{equilibrium} if the inequalities of Corollary \ref{thm:expected-equilibria-existence} are satisfied.

\begin{corollary}[Reverse engineering equilibria]

  Let $\lambda_i\in[0,1], i=1,\ldots,N$ be any vector of probabilities, and write
  %
  \[
    z_i = Q_{Z_i(\tilde\theta_i,\lambda_{-i})}(\lambda_i) \quad i=1,\ldots,N.
  \]
  %
  Then $(\lambda_i)_{i=1}^N$ is a Nash equilibrium of $(\Mechanism,\vec{z})$.
  %
  Moreover, $z$ is the lowest offer that can achieve a strategy profile at least as good as $\lambda$: if $\vec{z}'$ is any offer that admits an equilibrium $\lambda_i'$ with $\lambda_i'\geq\lambda_i$ for all $i$, then $z_i'\geq z_i$ for all $i$.

\end{corollary}
%
\begin{proof}

  By general properties of the quantile function. \qedhere

\end{proof}

\begin{corollary}[Upper bounds for the cost of censorship]

  For any $\lambda_i$,
  %
  \[
    \CC{\prod_{i=1}^N\lambda_i}(\Mechanism) \leq \sum_{i=1}^N Q_{Z_i(\tilde\theta_i,\lambda_{-i})}(\lambda_i) .
  \]
  %
  For any $p\in[0,1]$, we have
  %
  \[
    \CC{p}(\Mechanism) = \inf_{\prod_i\lambda_i \geq p} \left(\sum_{i=1}^N Q_{Z_i(\tilde\theta_i,\lambda_{-i})}(\lambda_i) \right).
  \]
  

\end{corollary}



We will distinguish two basic classes of examples:
\begin{enumerate}
  \item Common value, i.e.~the $v_i$ are deterministic and known to all participants;
  \item Strictly monotone, i.e.~the function $c_i(v)$ is strictly increasing in $v$ and the distribution of $\tilde{v}$ is without atoms.
\end{enumerate}

\paragraph{Common value case}
Here $\bar\lambda(z)=\lambda^*(z)$ and we can solve for $z$.
%
The special case $\tilde\theta \equiv 0$ is studied in \cite{FPR}.


\begin{example}[Second price common value auction]

  As we show in an earlier example, the surplus coefficients are $c_0 = v$ and $c_k=0$ for $k>0$.
  %
  If $z>v$, all bidders accept the bribe.
  %
  If $z\in[0,v]$, the bidders pay a mixed strategy defined by the equation $z=v\lambda^{N-1}$; that is,
  \[
    \lambda^*(z; v) = (z/v)^{1/(N-1)}.
  \] 
  We note that it is a strictly increasing and concave function of $z$, and strictly concave if $N>2$.
  %
  If $N=2$, we have $\lambda^*(z)=z/v$.
  
  Finally, we note here that the logarithmic derivative is 
  \[
    \frac{d\log \lambda^*}{dz}(z)= ((N-1)vz)^{-1}.
  \]

\end{example}

\begin{comment}

\begin{question}What happens when obtain large $N$ by adding a long tail of bidders with low $\tilde{v}$?\end{question}

\begin{example}[Bidder reputation]

  A general rule of thumb is that the more information the adversary has, the cheaper it becomes to bribe.
  %
  For example, consider the case that bidders are partitioned into two sets: a `high reputation' set $[N]$ of bidders that handle most uncensored requests, and a long tail 'low reputation' set $[M]$ that comes into play when the high reputation set has been bought off.
  %
  The high reputation set may be quite small; for example, $N=2$ in PBS today (over 90\% of blocks are handled by just two entities).

  Suppose that types and strategies are symmetric within each of the two sets, and that $v_N>v_M$ almost surely. 
  %
  We are allowed two bribe amounts $z_N$ and $z_M$ and strategic plans $\lambda_N^*(\tilde{v},z_N)$ and $\lambda_M^*(\tilde{v},z_M)$.
  %
  Censorship succeeds with probability $p=\bar\lambda_N(z_N)^N\bar\lambda_M(z_M)^M$.


\end{example}

\end{comment}






\end{document}
























\newpage
\section{Discussion}


Situations:
\begin{itemize}

  \item 
    The bribe vector $\vec{z}$ is known to all. The adversary must contract with each scheduler in isolation; commitments are binding even if only a subset of offers are accepted.

  \item
    The adversary may attempt to secure a single agreement involving all or a subset of schedulers.
    %
    The agreement is not binding unless all counterparties agree.
  
  \item
    The adversary attempts to secure an agreement as above, but governing multiple auctions for different items over time where different schedulers may be active in different epochs.

\end{itemize}

\subsection{Conditional contracting}

Here I'll make a brief comment on how costs could come down under certain forms of conditional contracts for exclusion:

\begin{itemize}
  \item \emph{ex-ante}. 
    Condition the terms of a group contract on everyone else joining the coalition. This effectively makes the construction of a coalition atomic.
    
    In terms of cost to censor, only the expected payoffs from the maximally competitive mechanism need be compensated.
    %
    In such cases, censorship resistance can actually \emph{fall} as $N$ increases (if $1/\Util_N$ grows faster than $N$).


  \item \emph{ex-post}.
    In the case of PBS: condition the payment on the builder winning the PBS auction. This could be effected by depositing funds in a smart contract and making them available to any entity who can prove that they built the target block and that the target block did not contain the targeted items.
    %
    This allows the censor to pay only one builder, though in this case the full revenue delta must be compensated instead of just the (expected) profit on $T$.

\end{itemize}






\subsection{Information asymmetry}

Some example scenarios I want to write about:
\begin{enumerate}
  \item
    Suppose, as in \cite{FPR}, that receipt of a bribe offer is taken as evidence that a censorship attempt is ongoing, with the assumption that the amount offered to other parties is similar.
    %
    Then a high bribe offer suggests a censorship that is likely to succeed, which also means that if the target agent rejects the bribery offer he is likely to face reduced competition and hence higher profits from the auction.
    %
    For this reason, symmetric bribes generally need to beat the expected \emph{monopoly} profit from the auction to have a high probability of success.

    However, a low but non-negligible probability of success can sometimes be achieved with a \emph{low} bribe.
    %
    Indeed, a bidder that ignores a low bribe is still likely to face some competition in the scheduler auction, so the bribe only needs to beat the \emph{competitive} profits (and monopoly profits with low probability).
    %
    We formalise this line of thinking in Example \ref{ex:non-negligible}.

  \item
    On the other hand, if a bribe is \emph{not} taken indicative of a censorship attempt that affects other bidders too, then more bidders can, apparently paradoxically, entail lower total bribery cost.
    %
    This is because the bribe only needs to beat competitive profits for each of $N$ bidders, which may be less than $1/N$ times the monopoly profits.

    However, I found it difficult to come up with any reasons that this model would be realistic.
    %
    One option is to have a long-lived adversary regularly issue `false flag' censorship bribes to individual bidders.
    %
    If bidders cannot see bribe offers extended to their competitors, then they cannot tell whether a real censorship attempt is underway.


\end{enumerate}


\begin{example}[Asymmetric offer]
  \label{asymmetric-offer}

  It is straightforward to describe a scenario in which an asymmetric offer achieves a lower cost of censorship at some quantile $p<1$ than a symmetric one.
  %
  Suppose $\Mechanism$ is a second price auction with reserve $\tau$ between two schedulers $a$ and $b$.
  %
  Ties are broken uniformly at random.
  %
  Suppose that the types of $a$ and $b$ are i.i.d.~with distribution function
  %
  \[
    F(t) = \left\{\begin{array}{ll}
      1-q & t<\tau \\
      1 & t \geq\tau
    \end{array}\right.
  \]
  %
  In other words, $\tilde\theta_a$ is $\tau$ with probability $q$ and $0$ otherwise, and similarly for $\tilde\theta_b$, and these events are independent.
  %
  We have
  %
  \[
    \Util_0(0) = \tau \qquad \Util_1(0) = q\tau  \qquad \Util_0(\tau) = \Util_1(\tau) = 0.
  \]
  %

  With probability $q^2$, both players have high costs and hence will accept any nonzero offer to abstain.
  %
  In order to achieve a censorship probability of greater than $q^2$, the offer must be acceptable to both players even when at least one has low costs.
  %
  A player with low costs will never accept an offer of less than $q\tau$ (regardless of the tie breaking rule), which brings the cost of censorship with a symmetric offer to at least $2q\tau$.

  On the other hand, if scheduler $a$ is offered an infinitesimal amount (call it $0$), then he accepts if and only if his costs are high, that is, with probability $q$.
  %
  In this case, if scheduler $b$ handles the item, he receives a profit of $\tau$ if $a$ has high costs and $0$ otherwise.
  %
  That is, the offer of $z_a$ to $a$ has not affected the expected payoffs of $b$, which remain $q\tau$ if his costs are low and $0$ otherwise, whence $b$ will accept any offer in excess of $q\tau$.

  Thus by offering $(0,q\tau)$, the adversary achieves a total censorship probability of $q>q^2$ with a total offer of $q\tau$, which is always less than the lowest feasible symmetric offer of $2q\tau$.
  %
  It follows that for any $p\leq q$, $\CC{p}(\Mechanism)\leq q\tau$, but for $q\in(q^2,q]$, the bound cannot be achieved with a symmetric offer.

\end{example}



%%%%%%%%%%%%%%%%%%%%%%%%%%%%%%%%%%%%%%%%%%%%%%%%%%%%%%%%%%%%%%%%%%%%%%%%%%%%%%



\subsection{Long-running coalition}
%
A similar formula comes up in the following more believable scenario: a proportion $\lambda\in[0,1]$ of the full scheduler population enters into a long-running coalition to censor items from a class $T$.
%
Assume that items are scheduled by a sequence of repeated auctions, one per epoch, in which $N$ members of the population are randomly sampled to participate.
%
Then the per epoch opportunity cost to a $\theta_i$-typed member of the coalition, and therefore the amount that must be compensated by the leader of the coalition, is
\[
  \lambda_i\sum_{k=1}^{N}\lambda^{N-k}(1-\lambda)^{k-1} \Expectation_i[c_k(\tilde\theta_{i,t})],
\]
%
where $\lambda_i$ is the member's election weight.
%
Censorship succeeds in a proportion $\lambda^N$ of epochs.
%
We have the structural equation
\[
  \CC{\lambda^N}(\Mechanism) = \RSGF_N(\lambda)
\]

Another, perhaps more immediately relevant, way that offers can arise is in the form of threats, for example from regulators (OFAC) or legal contracts.
%
A legally binding commitment or exposure of this type encumbers the affected agent with a \emph{risk cost} $z$ for breaching (for example, an added insurance premium or legal representation cost).
%
Such exposures have the following characteristics:
\begin{itemize}
  \item They asymmetrically affect agents in different jurisdictions; there is no reason to expect all bidders to be affected by one regulatory system.
  \item They are long-lived.
  \item They tend to be public. (In the case of some legal contracts such as non-compete agreements, they may not be public, but this example is rather speculative.)
\end{itemize}


For the next result, we shift focus from a single transaction item to a long-running arrangement where a class of transactions are to be dropped in every slot in which the participating schedulers are active.
%
We assume that negotiation is possible and that the set of schedulers who join the coalition is fixed throughout its duration.

\begin{theorem}[Cost to form a long-running censoring coalition]

  The cost to form the coalition $\CC{p}(\Mechanism)$ is a degree $N$ polynomial in $q=p^{1/N}$ with positive real coefficients satisfying the differential relation
  \[
    0= \frac{\partial}{\partial q}\CC{q^N}(\Mechanism) + \sum_{k=1}^N {N-1 \choose k-1} F_{\tilde\theta}(q)^{k-1} q^{N-k}(1-q)^k.
  \]

\end{theorem}


\begin{example}[Second price common value]

  We return to the case of the second price auction with common value $v$.
  %
  Recall that $\lambda^*(z)=(z/v)^{1/(N-1)}$ on $[0,v]$.
  %
  Then 
  \begin{align*}
    U_\Adversary(z) &= v_\Adversary (z/v)^{\frac{N}{N-1}} - Nz(z/v)^{\frac{1}{N-1}} \\
    &= z^{\frac{N}{N-1}}v^{-\frac{1}{N-1}}\cdot \left(v_\Adversary/v - N  \right)
  \end{align*}
  for $z\in[0,v]$.
  %
  This utility function is concave (resp.~convex) if and only if it is negative (resp.~positive) for $z>0$.
  %
  The auction is censorship resistant if and only if $v_\Adversary < Nv$ (concave).

  In the case of a procurement auction with max fee $\pi$ and common marginal cost $\mu$ \cite[\S3.2]{roughgarden2024transaction}, the bound becomes $N(\pi-\mu)$.

\end{example}

\begin{comment}
\newpage
\section{Dynamic scheduler population}


\paragraph{Dynamic bidder populations: implementation notes}
In applications, the set of entities authorised to allocate blockspace is listed in some kind of registry.
%
For example:
\begin{itemize}
  \item 
    The set of entities that may propose blocks in Ethereum conensus is indexed by a set 20-byte Ethereum addresses listed in a stake registry. 
    %
    This registry may be updated with the additional or removal of entities.

  \item
    Block builders in PBS must register with a relay to participate in the auction.

  \item
    Preconfirmation providers or `preconfers' in various models of preconfirmations (a type of blockspace forward in which a specific transaction or transaction bundle is guaranteed to land in a certain block) must insert themselves into a registry and put up collateral for their commitments.
\end{itemize}
%
The implementation of the registry is not important for our purposes; as well as a centralised (in the case of the PBS relays) or on-chain (in the case of proposers or preconfers) database, blockspace allocators could simply be the holders of a local authorisation token.

\begin{itemize}
  \item 
    Limits on implementation in smart contracts mean that such mechanisms should have simple rules (so no optimal stopping based on statistical models, yes clock auctions).
  
  \item
    Increasing the number of schedulers can result in increasing computational \cite{buterin2017parametrizing} or consensus \cite{wang2023security} difficulties.
  
  \item
    Certain types of computational tradeoffs are well-studied in the mechanism design literature [cite something about clock auctions?]
    
  \item
    From another perspective, dynamic auctions allow us to posit \emph{dynamic populations} that can expand and contract in response to new information in the market --- for example, information about order flow that has remained unclaimed for some time and hence may be the target of a censorship attempt.
\end{itemize}




\begin{itemize}
  \item
    A scheduler competition where different schedulers arrive over time facilitates the spread of information about other bidders' preferences, and hence in particular, whether censorship attempts may be occurring.

  \item
    So even in an environment where schedulers cannot easily infer the general strategy of an adversary purely from a bribe offered to them, information about that strategy can nonetheless be revealed by the inaction of other schedulers.

  \item
    Sequences of secondary markets for blockspace (resp.~orderflow) should surface price information about that blockspace (orderflow).
    %
    A censorship attempt negatively affects the market price of the target orderflow items.

  \item
    Constructing a grand coalition requires knowledge of the complete scheduler set at the time of contracting (or at the very least, at the earliest time that coalition members could deviate).
    %
    Dynamic scheduler populations undermine this approach; new contracts must be agreed as new schedulers arrive.
    %
    Examples include proposers not in the lookahead, or schedulers taken from a large population eligible to acquire and exercise a transferable blockspace future.

  \item
    There may also be other practical reasons to prefer dynamic scheduler populations, for example messaging complexity.
    %
    A large number of always-active schedulers may have negative implications for throughput or consensus stability \cite{wang2023security}; a small number of always-active schedulers with a larger number sitting in reserve that come into action during a period of market failure could achieve the same efficiency goals without the tradeoffs.

  \item
    For example, in the case of sequentially arriving, short-lived bidders, where contracting with a bidder is not possible before his active period, the product formula bound $\sum_{i=0}^{N-1}(v_i-v^*)$ is realised.
    %
    In this case, it's clearly in the would-be scheduler's interest to remain non-addressable at first, so that the grand coalition contract cannot be conditioned on his behaviour.
    
    Low-cost entry to the scheduler market is key to this result.

\end{itemize}


\begin{itemize}
  \item 
    If we find anything here, it should be that the CR calculations are much simpler and more convincing than in the parallel case.
  \item
    The game sequence where schedulers first are declared active and then enter into contracts to become inactive ($\in\Population_\delta$) again may seem convoluted, but we will argue that this actually most resembles a realistic situation in the presence of markets of secondary holders of transferable blockspace future.

  \item
    In fact, it already somewhat resembles the situation with proposers outside the lookahead, or with miners in a PoW blockchain like Bitcoin.
    %
    These proposers do not directly control their entry into the active set $\Population_\alpha$.

  \item
    For simplicity we assume equal weighted bribes are used, but in reality this approach is probably suboptimal for the asymmetric sequential arrivals model.
    %
    Some cooperative game theory should be used to do it properly.
  \item
    The focus on posted price mechanisms (including with a `fire sale' at the end) is not artificial: it is known that in the cases of interest, these implement efficient allocations.
\end{itemize}

We now anchor ourselves in a dynamic setting in which would-be schedulers must first declare themselves as \emph{active}, incurring a cost $\kappa>0$, before they may participate in the blockspace supply mechanism --- whether by bidding to schedule items or by entering into contracts to abstain.
%
The scheduler population $\Population$ may be infinite, but only finitely many schedulers may declare themselves active at any one time.
%
Denote by $\Population_\alpha$ the active population.

Each would-be scheduler decides whether to declare themselves active, given information available at time $t$, if their expected surplus from participation exceeds the registration cost, i.e.~if
\[
  c_t(v) > \kappa.
\]
We suppose it becomes common knowledge when an item has been scheduled.

\begin{example}[Sequential arrival of short-lived schedulers]

  Suppose that exactly one scheduler may declare itself active at each epoch.
  %
  At the end of the epoch, the active scheduler becomes permanently inactive, i.e.~enters $\Population_\delta$.
  %
  Thus schedulers do not need to consider the value of withholding blockspace for allocation at a later epoch.
  
  Items are advertised with a posted fee $v^*(t)$ (which may depend on the epoch).
  %
  In each epoch, if a scheduler $a$ is active and $v_a<v^*(t)$, the item is allocated to $a$, who receives the fee $v^*$ and retains surplus $v^*-v_a$.
  %
  The (deterministic) surplus for the scheduler that may enter in epoch $t$ is
  \[
    c_t(v) = v^*(t)-v_a
  \]
  if $v^*(t)<v_a$ and $0$ otherwise.
  %
  Entry is feasible iff $v^*(t)-v_a > \kappa$.

  The censor's only strategy is to contract directly with the scheduler in epoch $t$ when it appears at the start of each epoch, at which point the price to beat is $v^*(t)-\tilde{v}_a$.
  %
  To achieve a probability $p>0$ of successful censorship over $N$ epochs, the adversary must pay
  \[
    z(t) = v^*(t) - F_{\tilde{v}}^{-1}(p^{1/N})
  \]
  to the scheduler in epoch $t$, where $F_{\tilde{v}}^{-1}$ is the quantile function of $\tilde{v}$.
  %
  If the posted price $v^*$ is constant, the total cost simplifies to
  \[
    CR = N(v^*-F_{\tilde{v}}^{-1}(p^{1/N})).
  \]
  
\end{example}

\begin{example}[Sequential arrival of short-lived schedulers, static population]

  Suppose allocation works as in the above example, except that the population of schedulers and their assignment to epochs is common knowledge at the start of play.
  %
  Then the $n$th scheduler has an expected surplus of
  %
  \begin{align*}
    c_n(v) &= \Prob[\tilde{v}_k < v^*, k=0,\ldots,n-1] \cdot (v^* - v) \\
    &= F_{\tilde{v}}(v^*)^{n-1}\cdot (v^* - v).
  \end{align*}
  %
  The total cost to subsidise a grand coalition with probability $p$ of success (with symmetric payouts) is
  \begin{align*}
    CR &= \sum_{n=0}^{N-1} F_{\tilde{v}}(v^*)^{n}\cdot (v^* - F_{\tilde{v}}^{-1}(p^{1/N})) \\
    &= \frac{1- F_{\tilde{v}}(v^*)^N}{1-F_{\tilde{v}}(v^*)} (v^* - F_{\tilde{v}}^{-1}(p^{1/N})).
  \end{align*}
  %
  This number is bounded above by $1/(1-F_{\tilde{v}}(v^*))$ as $N\rightarrow\infty$.
  %
  For example, if $v^*$ is chosen at the median of the type distribution so that $F(v^*)=1/2$, the maximum censorship resistance factor is just $2$.

  Now suppose the arriving schedulers know their assignment to epochs in advance, and may \emph{choose} whether or not to make themselves publicly known at the start of the period.
  %
  If $v_\Adversary > N(v^* - v(p^{1/N}))$, then the scheduler can arrange a larger bribe for himself by choosing not to reveal himself until the epoch in which he becomes active.
  %
  On the other hand, if the adversary's budget would cover a grand coalition on a static population declared at the start but not the full cost of a dynamic approach, the scheduler at the back of the queue may be better off declaring himself in advance.

\end{example}

\begin{example}[Sequential arrival of long-lived schedulers]

  If bidders are long-lived, they may have multiple opportunities to bid for an item.
  %
  Conversely, prospective new entrants must consider whether they will face competition from incumbent schedulers who haven't yet received an allocation.

  If the item is not scheduled in a given epoch $t$, it may be inferred that no scheduler with cost $v_i<v^*$ was active during that epoch.
  %
  It follows that $\Population_{\alpha,t} = \Population_{\delta,t}$, and as-yet-inactive members of $\Population$ may estimate their entry surplus $c_{t+1}(v)$ as though they are entering a pristine competition environment.

  Suppose that registration rates are throttled so that at most $N$ new schedulers may register at any time.
  %
  (How are the $N$ new schedulers selected? We haven't allowed them to differentiate based on registration cost $\kappa$.)
  %
  The short answer is that the more schedulers get selected

\end{example}

\end{comment}


\newpage
\appendix

\section{Pointed mechanisms}


\subsection{Pointed mechanisms}

We will study procurement auctions $\mathcal{M}$ with $N$ single parameter bidders and allocation and payment functions 
\[
  x: \uR^N\rightarrow [N] \qquad \pi:\uR^N\rightarrow \uR^n.
\]
%
A key feature is that in addition to any other actions they may take, bidders always have a choice to \emph{abstain} from bidding, regardless of their type.

\begin{definition}

  A \emph{pointed mechanism} is a mechanism $\Mechanism=(\Action,\Theta,U:\Action\times\Theta\rightarrow\R^N)$ each of whose strategy spaces $\Action_i$ are equipped with a distinguished point $\bot\in\Action$ such that
  \[
    U_i(\bot,s_{-i})=0 
  \]
  for all competitor strategy profiles $s_{-i}$ and $s_i$.
  %
  The strategy $\bot$ is called \emph{abstention}.

  If $i\in[N]$, then the \emph{complement} of $i$ in $\mathcal{M}$ is the mechanism $\mathcal{M}_{-i}=(\prod_{j\neq i}\Action_j,\prod_{j\neq i}\Theta_j,V)$ where $V_j$ is defined by the formula
  \[
    V_j(s,\theta_j) = U_j((s,\bot),\theta_j)
  \]
  for each $j\neq i$.
  
  A \emph{compatible sequence} of pointed mechanisms is a sequence $(\mathcal{M}_N)_{N\in\N}$ together with isomorphisms $\mathcal{M}_{N-1}\cong (\mathcal{M}_N)_{-N}$ for each $N$.
  %
  This notion is needed to study large $N$ asymptotics of mechanisms (without having to deal with infinite products).

\end{definition}

Any mechanism can be canonically transformed into a pointed mechanism by formally adding an element $\bot$ to the action space and assigning the utility $U_i(\bot,b_{-i})=0$ for all strategy profiles $b_{-i}\in\Action_{-i}$.

The strategy space $\Action=\prod_i\Action_i$ of a pointed mechanism comes with a function $\Population_\Active(s):\Action\rightarrow 2^\Population$ that picks out the \emph{active set}, that is, the set of participants that did not choose to abstain.

\begin{example}[Direct revelation]

  A direct revelation mechanism has action space equal to the type space $\Theta_i$.
  %
  It can be transformed into a pointed mechanism with strategy space $\Theta_i\sqcup\{\bot\}$.
  %
  Note that some direct revelation mechanisms may already admit a pointing, such as $0\in [0,R]\simeq\Theta_i$.

\end{example}

\begin{definition}[Competitiveness]
  \label{def-competitive}

  A pointed direct revelation mechanism $U:\prod_i\Theta_i\sqcup\{\bot\}\rightarrow\R^\Population$ is \emph{competitive} if the inequality
  %
  \[
    \Util_j(\bot_i,\theta_{-i}) \geq \Util_j(\theta_i,\theta_{-i})
  \]
  %
  for every $\theta_i\in\Theta_i$, $\theta_{-i}\in\Theta_{-i}$.

\end{definition}

Suppose $\Theta_i\subseteq\R$.
%
A (direct revelation) selection mechanism is \emph{efficient} if for all $\theta\in\overline{\Theta}$, $\theta_{x(\theta)}$ is maximal among the active coefficients of $\theta$.
%
Every efficient mechanism is competitive.

\begin{definition}[Single scheduler selection]

  A \emph{single scheduler selection mechanism} is a tuple $\Mechanism=(\Population, \tilde\theta, \Action=\prod_{i\in\Population}\Action_i,x:\Action\rightarrow\Population,f:\Action\rightarrow\uR^\Population)$, where
  \begin{enumerate}
    \item $\Population$ is a finite set;
    \item $F_{\tilde\theta}$ is a \emph{cost} distribution on $\uR^\Population$;
    \item $\Action_i$ is a pointed set;
    \item $x(s) \neq i$ if $s_i=\bot$;
    \item $f(s) = 0$ if $s_i=\bot$.
  \end{enumerate}
  %
  We also assume that the total fee payment is bounded above by some \emph{max fee} $\tau>0$, i.e.~$\sum_if_i(s)\leq\tau$.
  
  We associate a pointed mechanism to a single scheduler selection mechanism by setting $\Theta_i=\uR$ and
  \[
    U_i(s,\theta_i) = f(s) - I_{x(s)=i}\theta_i.
  \]
  %
  If $j\in \Population$, we can define the \emph{complement of $j$} as the selection mechanism $(\Population\setminus\{j\},\tilde\theta,\prod_{i\neq j}\Action_i,x',f')$ where $x'(s_{-j}) = x(\bot,s_{-j})$ and $f(s_{-j})=f(\bot,s_{-j})$; the hypotheses on $x$ and $f$ ensure that these belong to $\Population\setminus\{j\}\subset\Population$ and $\Action_{-j}\simeq \Action_{-j}\times\{\bot\}\subset \Action$, respectively.

  A necessary condition for competitiveness of the associated pointed mechanism is that there is no strategy profile $s$ such that $f_i(s)>0$ and $x(s)=0$.
  %
  The competitiveness condition is
  %
  \[
    \Prob(x(s)=i \mid s_j = \bot) \geq \Prob(x(s)=i \mid s_j\neq\bot)
  \]
  %
  for all $i\neq j$.

\end{definition}


\subsection{Conditional values}

\begin{definition}

  The \emph{conditional value} of player $i$ with active set $S\subseteq \Population$ is
  \[
    \Util_{i,S}(\theta) = \Expectation[\Util_i(\tilde{\theta}) \mid \Population_\alpha(s)=S].
  \]
  %
  In the case of a single scheduler selection mechanism $x:\Action\rightarrow\Population$, $f:\Action\rightarrow\uR^\Population$ with quasi-linear payoffs, by Myerson's Lemma we have
  %
  \[
    \Util_{i,S}(\theta) = \Util_{i,S}(\theta') + \int_{\theta}^{\theta'} \Prob[i \text{ wins} \mid \tilde{\theta}_i = s,\,\Population_\alpha = S] ds.
  \]
  %
  Suppose that the item offers a maximum fee $\tau$, so that $\sum_if_i(\vec{s}) \leq\tau$.
  %
  Then $\Util_{i,S}(\theta) \leq [\tau - \theta]_+$ in all cases, and so
  %
  \[
    \Util_{i,S}(\theta) = \int_{\theta}^\tau \Prob[i \text{ wins} \mid \tilde{\theta}_i = s,\,\Population_\alpha = S] ds.
  \]


  If the mechanism is symmetric, then this function is independent of $i$ and depends on $S$ only through its cardinality; we write
  \[
    \Util_{n}(v) = \Expectation[\Util_i(v) \mid \Population_\alpha(\theta_{-i})=S]
  \]
  for $n\in\N$ and any $S$ with cardinality $n$.
  
\end{definition}


\begin{example}[Vickrey auction with uniform type distribution]

  Suppose $\Mechanism$ is a second price auction with bidder types uniformly distributed on $[0,1]$.
  %
  Then 
  \[
    \Util_k(\theta) = \theta \left( 1 - \frac{k\theta^k}{k+1} \right)
  \]
  for all $k$ and $\theta\in[0,1]$.
  
  One can show that the coefficients of $Z(p,1-p)(1)$ are all positive.

\end{example}

\begin{lemma}[Monotonicity of conditional values]
  \label{conditional-value-monotonic}

  Suppose that $\Mechanism$ is competitive.
  %
  Then for any $i,j\in\Population$ with $i\neq j$, $S\subseteq\Population\setminus\{i,j\}$, and $\theta_i\in\Theta_i$, the inequality
  \[
    \Util_{i,S}(\theta_i) \geq \Util_{i,S\sqcup\{j\}}(\theta_i)
  \]
  %
  is satisfied.

\end{lemma}
%
\begin{proof}

  Immediate from the definition \ref{def-competitive} of competitiveness. \qedhere

\end{proof}

\begin{proposition}[Surplus equations for efficient mechanisms]

  If $\Mechanism$ is an efficient auction with max fee $\tau$, then
  \[
    \frac{\partial \Util_k}{\partial\theta} ( \theta ) = -F_{\tilde{\theta}}(\theta)^k
  \]
  for all $k\geq 0$, where $F_{\tilde{\theta}}$ is the CDF of the cost distribution.

\end{proposition}

\begin{example}[Single item auction]

  In the focus case of a single item auction (where the item cannot be awarded to any abstaining bidder), we have
  \[
    c_{i,S}(v) = \Prob[x(b)=i\mid \Active(b_{-i})=S]\cdot v - \Expectation[\pi_i(b)\mid \Active(b_{-i})=S].
  \]
  A procurement auction is handled the same way, except the valuation $v<0$ represents the \emph{cost} of providing the service and payments $\pi_i<0$ go the other way.

\end{example}

It's worth singling out the coefficient $c_0(v)=c_\emptyset(v)$ as the \emph{monopolist} case.
%
In an efficient single item procurement auction, this number is $f_{\max}-v$ where $f_{\max}$ is the auction \emph{reserve price}, that is, the maximum fee the auctioneer is prepared to pay.

The \emph{gap} $c_0-c_1$ may be substantial.

Moreover, by revenue equivalence we have $c_0 \geq c_1 \geq \cdots \geq c_{N-1}$ for any mechanism that implements the efficient allocation, with all inequalities strict if the bidder type distribution is atom free.



\begin{lemma}

  For any profile of optimal strategic plans, $\Util_i(\theta_i,z_{-i})$ is monotone decreasing in all arguments.

\end{lemma}
%
\begin{proof}

  By Myerson's Lemma \cite[Thm.~3.3]{milgrom2004putting}, each coefficient $c_{i,S}(\theta_i)$ is monotone decreasing in $\theta_i$, while the monomials $\bar\lambda_j(z_j)$ are independent of $\theta_i$.
  %
  Meanwhile, $c_{i,S}(\theta_i)$ is independent of $z_j$ by construction, while $\bar\lambda_j(\vec{z})$ is increasing in $z_j$.
  %
  Write
  \begin{align*}
    c_i(\theta_i,z_{-i}) =& \bar\lambda_j(z_j) \sum_{S\ni j} c_{i,S}(\theta_i) \prod_{r\in S\setminus j}\bar\lambda_r(z_r) \prod_{i\neq k\not\in S} (1-\bar\lambda_k(z_k)) \\ 
    &+ (1-\bar\lambda_j(z_j))\sum_{S\not\ni j} c_{i,S}(\theta_i) \prod_{r\in S}\bar\lambda_r(z_r) \prod_{k\in\Population\setminus(S\sqcup\{i,j\})} (1-\bar\lambda_k(z_k)).
  \end{align*}
  %
  Increasing $\bar\lambda_j(z_j)$ moves weight from the second term to the first term, so the result will follow from the inequality
  %
  \[
    \sum_{S\ni j} c_{i,S}(\theta_i) \prod_{r\in S\setminus j}\bar\lambda_r(z_r) \prod_{k\in\Population\setminus\{i\}} (1-\bar\lambda_k(z_k)) \\ 
    <\sum_{S\not\ni j} c_{i,S}(\theta_i) \prod_{r\in S}\bar\lambda_r(z_r) \prod_{k\in\Population\setminus(S\sqcup\{i,j\})} (1-\bar\lambda_k(z_k))
  \]
  %
  which can be shown by summing the inequality
  %
  \[
    c_{i,S}(\theta_i) \geq c_{i,S\sqcup\{j\}}(\theta_i)
  \]
  %
  (Lemma \ref{conditional-value-monotonic}) over $S\subset \Population\setminus\{i,j\}$. \qedhere

\end{proof}



\subsection{Payoff generating function}


These assemble into a \emph{payoff generating function}
\[
  Z(p,q;v) = \sum_{k=0}^N c_{k}(v){N\choose k} q^kp^{N-k}.
\]
For fixed $v$, $Z(p,q;v)$ is a polynomial in $p$ and $q$.

\begin{lemma} $\frac{\partial Z_N}{\partial p} = NZ_{N-1}$ \end{lemma}

\begin{lemma}[Positivity of coefficients]

  The coefficients of $\frac{\partial}{\partial p}Z(p,1-p)$ are all non-negative.
  %
  If $\mathcal{M}$ is gapped, then at least one coefficient is positive.
  %
  The same goes for $\frac{\partial^2}{\partial p^2}Z(p,1-p)$.
  %
  In particular, $Z(p,1-p)$ is smoothly invertible and convex on $[0,1]$.

\end{lemma}
%
\begin{proof}

  It's enough to show that the coefficients of $(\partial/\partial p - \partial/\partial q)Z(p,q)$ are all positive.
  %
  We have
  \begin{align*}
    \frac{\partial Z}{\partial p}(p,q) &= \sum_{k=0}^{N-1} c_k {N \choose k}(N-k)p^{N-k-1}q^k \\
    \frac{\partial Z}{\partial q}(p,q) &= \sum_{k=1}^N c_k {N \choose k}kp^{N-k}q^{k-1} \\
    &= \sum_{k=0}^{N-1} c_{k+1} {N \choose k+1}(k+1) p^{N-k-1}q^{k}.
  \end{align*}
  So the coefficient of $p^{N-k-1}q^k$ is given by
  \[
    c_k {N\choose k}(N-k) - c_{k+1} {N\choose k+1}  (k+1) =  \frac{N!}{k!(N-k-1)!} (c_k-c_{k+1}) > 0. 
  \]
  It follows similarly that the coefficient of $p^{N-k-2}q^k$ in the second derivative is $\frac{N!}{k!(N-k-2)!}(c_k-c_{k+2})\geq 0$.

\end{proof}

\paragraph{Inverse of $Z(p,1-p)$}

Differentiating, we have
\begin{align*}
  Z'(q) &= -c_0(N-1)(1-q)^{N-2} + c_1(N-1)(1-q)^{N-2} + O(q) \\
  &= (N-1)(1-q)^{N-2}(c_1-c_0) + O(q) \\
  &= (N-1)(c_1-c_0) + O(q)
\end{align*}
which is strictly negative in a neighbourhood of $q=0$; unsurprising, since we expect that lowering the offer increases the probability of participation.
%
Therefore $Z(q)$ has an analytic inverse $q^*=G(z)$ near $q=0$, and 
\[
  G'(Z(q)) = 1/F'(q) \approx \frac{1}{(N-1)(c_1-c_0)}.
\]

\subsection{Random inequalities I haven't been able to use}

\begin{itemize}

  \item If $Q$ is the quantile function of $F$, then $Q(F(z))\leq z$ and $F(Q(p))\geq p$.
  \item Quantile functions are additive over non-negative, monotone increasing functions of a random variable: $Q_{\Util(\tilde\theta)} = \sum_{k=1}^N Q_{\Util_{k-1}(\tilde\theta)} \cdot stuff$.
  \item For an efficient single item auction, $Q_{\Util_0(\tilde\theta)} = \tau + Q_{-\tilde\theta}$.
  \item But the (infimum) quantile function doesn't make sense in the multivariate case.

\end{itemize}


\begin{lemma}

  \label{thm:quantile-sum}

  We have $Q_{\Util(\tilde\theta,\vec{z})} = \sum_{k=1}^N { N-1 \choose k-1 } Q_{\Util_{k-1}(\tilde\theta)} \bar\lambda(\vec{z})^{N-k} (1-\bar\lambda(\vec{z}))^{k-1}$.

\end{lemma}
%
\begin{proof}

  Because each $\Util_k$ is a monotone decreasing function of $\tilde\theta$, each value of $\Util(\tilde\theta,\vec{z})$ is achieved at the same time as at most one set of values of $(\Util_k(\tilde\theta))_{k=0}^{N-1}$. \qedhere

\end{proof}


\begin{lemma}

  Let $\tilde\theta$ be a real valued random variable and let $g:\R\rightarrow\R$ be a monotone decreasing function defined on the support of $\tilde\theta$.
  %
  Then for any $p\in[0,1]$, $g(\hat Q_{\tilde\theta}(1-p)) \geq Q_{g(\tilde\theta)}(p)$, where $\hat Q_{\tilde\theta} = -Q_{-\tilde\theta}$ is the `lower quantile' function of $\tilde\theta$.

\end{lemma}

\printbibliography